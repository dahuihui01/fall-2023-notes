\chapter{Preface}
\section{Lecture 1}
Here we go.
\subsection{Logistics}
\textbf{OH}: Wednesday 1-2pm virtual, 2-4pm Evans 813.

\textbf{Textbook}: Fourier Analysis by J. Duoandikoetxea (plan to cover chapter 1-6, and sections 1-4 of chapter 8); other texts: \textit{Introduction to Fourier Analysis on Euclidean Spaces} by Stein and Weiss, \textit{Singular Integrals and Differential Properties of Functions} by Stein, \textit{Harmonic Analysis} by Stein.

\textbf{Topics}: Fourier series, Fourier transform, maximal functions, Hilbert transform, singular integrals, Littlewood-Paley theorem, multipliers, oscillatory integrals

\textbf{Grading}: The grading will be entirely dependent on 3 problem sets given throughout the semester (with an ample amount of optional problems).


\subsection{Course Overview}
We will begin by defining what ``Harmonic'' means in the context of Math 258: to us, this word harmonic refers to "Euclidean Fourier analysis." And more specifically, we will study Fourier analysis on the $n$-dimensional torus, in $\R^n$. One justification for studying on/in these spaces is that many are equipped with translation invariance, which among other things, gives us nice behaving eigenfunctions.

Consider the function $e(x):=e^{2\pi inx}$, and consider the translation operator $f_t(x)=f(x+t)$, we have
\begin{equation*}
    f_t(e(x))=e^{2\pi in(x+t)}=e^{2\pi int}\cdot e^{2\pi inx}
\end{equation*}

Here, $e^{2\pi inx}$ can be seen as an eigenfunction of translations. Another obvious example is differentiation. Consider the differentiation operator on $e(x)$, we have
\begin{equation*}
    \partial_x(e(x))=2\pi in(e^{2\pi inx})
\end{equation*}

Again, $e^{2\pi inx}$ is an eigenfunction. In the forseeable future, we will see $\{e^{2\pi inx}\}_{n\in\mathbb{N}}$ forms a basis of funcitons on the 1-dim torus, $\mathbb{T}=\R/\mathbb{Z}$ i.e. functions on the circle. Likewise, we have $\{e^{2\pi i\sum n_ix_i}\}_{n_i\in\mathbb{Z}}$ as the basis of functions in the $n$-dim torus, defined as $\mathbb{T}^n=(\R/\mathbb{Z})^n$. They have the nice properties of diagonalizing translation, differentiation operators as they are eigenfunctions. Similarly, we have $\{e^{2\pi i\sum n_ix_i}\}_{n_i\in\mathbb{Z}}$ for $\R^n$, and we say they are ``almost in $L^2$,'' or $L^2$-wannabes as they are not far from $L^2$, but not quite in $L^2(\R^n)$.

\begin{remark}
    This property gives them the importance of in studying differential operators with constant coefficients
\end{remark}

We will go through various technical things along the way, one of them being ``cancellation.'' In the most general sense, using triangle inequality for everything is quite of a waste, for example, for highly oscillatory functions. We would like to exploit whenever we can, such as the oscillations of functions, kernels of operators, etc. More importantly, we will use different methods for different parts, to treat different issues. In other words, one should go to the dentist when they broke their ankle.

We will  first study the question when do partial sums of a Fourier series (of functions on the cirlce) or Fourier transform of functions in the Euclidean space converge, and converge in what sense. Convergence usually has two ''senses:'' pointwise convergence and $L^p$ norm convergence. We will study both.

\newpage
\chapter{Fourier Series and Fourier transform}
\section{Lecture 2}
\textbf{More logistics}: OH's have been updated as follows: Wednesday 10:30am-11:30am, 2-3pm.

We now begin Chapter 1 of our text.

Recall we define the 1-d torus as $\mathbb{T}=\R/\mathbb{Z}$, and the functions on the torus are naturally identified with the functions on the unit interval. Fourier analysis began when Fourier asked the following question: given a function $f$ on the circle, can we find $a_k, b_k$ such that the following is true:
\begin{equation*}
    f(x)=\sum_{k=0}^\infty a_k\cos(2\pi kx)+b_k\sin(2\pi kx)
\end{equation*}

The modern Fourier analysis asks the following, can we find $c_k$ such the following is true:
\begin{equation*}
    f(x)=\sum_{k=0}^\infty c_ke^{2\pi ikx}
\end{equation*}

The above two questions are identical if we take $a_k=c_k, b_k=ic_k$.

One intuitioin for having $2\pi k$ in the $\cos, \sin$ is that we would like to have periodic functions with period 1 to approximate $f$. Now we introduce the Fourier coefficients, and we first motivate this using trigonometric polynomials of the form 
\begin{equation*}
    f(x)=\sum_{k=0}^N c_ke^{2\pi ikx}
\end{equation*}

We only know $f$ is a finite sum of $e^{2\pi ikx}$, yet we would like to know the $c_k$'s. And we do this by exploiting the orthogonality of $\{e^{2\pi ikx}\}$. Notice we have the following:
\begin{equation*}
    \int_0^1 e^{2\pi ik_1x}\overline{e^{2\pi ik_2x}}dx=\begin{cases}
        1, k_1=k_2\\
        0, k_1\neq k_2\\
    \end{cases}
\end{equation*}

We therefore have, for any fixed $k$, 
\begin{equation*}
    \int_0^1f(x)e^{-2\pi ikx}dx=\int_0^1\left(\sum_{k=0}^nc_ke^{2\pi ikx}\right)e^{-2\pi ikx}dx=c_k
\end{equation*}


\begin{definition}[Fourier coefficients, Fourier series]
    Given $f\in L^1(\mathbb{T})$, or any periodic function on $\R$ with period 1 that is locally integrable, we define its $k$-th Fourier coefficient as follows:
    \begin{equation*}
        \hat{f}(k)=\int_0^1f(x)e^{-2\pi ixk}dx
    \end{equation*}
    Given the Fourier coefficients, we write $f$'s Fourier series as follows:
    \begin{equation*}
        f(x)\sim \sum_{k\in\mathbb{Z}}\hat{f}(k)e^{2\pi ikx}
    \end{equation*}
\end{definition}

For trigonometric polynomials, as we saw, its Fourier series agrees exactly with itself, and its Fourier series will only have finitely many terms. However, more often than not, arbitrary $f$'s Fourier series will have infinitely many terms, and now we go back to the question: when will $f$'s Fourier series converge and converge in what sense.

We define partial sums of Fourier series as $S_N(f)(x)=\sum_{|k|\leq N}\hat{f}e^{2\pi ikx}$, and pointwise convergence asks the question: for fixed $x$, when do we have
\begin{equation*}
    \lim_{N\to\infty}S_N(f)(x)=f(x)
\end{equation*}

We now introduce two theorems on pointwise convergence (that will be proved in the next lecture).
\begin{theorem}[Dini's criterion]
    Fix $x\in\mathbb{T}$, if we have
    \begin{equation*}
        \int_{|t|<\delta}\left|\frac{f(x+t)f(x)}{t} \right|<\infty
    \end{equation*}
    Then we have
    \begin{equation*}
        \lim_{N\to\infty}S_N(f)(x)=f(x)
    \end{equation*}
\end{theorem}

For example, Lipshitz functions would have their Fourier series pointwisely converge. More generally, if a function blows up around a point only slightly, then we would have its Fourier series converge to it at that point.

\begin{theorem}[Jordan's criterion]
    Fix $x\in\mathbb{T}$, and some $\delta>0$, if $f$ is of bounded variation in $(x-\delta, x+\delta)$, then we have
    \begin{equation*}
        \lim_{N\to\infty}S_N(f)(x)=f(x)
    \end{equation*}
\end{theorem}

Again, we delay the proof till next time.

To study pointwise convergence, we first note that $S_N(f)$ is a convolution operator.
\begin{align*}
    S_N(f)(x)&=\sum_{|k|\leq N}\hat{f}(k)e^{2\pi ikx}\\
    &=\sum_k\left(\int_0^1f(t)e^{-2\pi ikt}dt \right)e^{2\pi ikx}\\
    &=\int_0^1f(t)\sum_ke^{2\pi ik(x-t)}dt\\
    &=f(x)\ast D_N(x)
\end{align*}
where $D_N(x)=\sum_{|k|\leq N}e^{2\pi ikx}$, and this kernel is called the Dirichlet kernel.

\begin{remark}
    To study the pointwise convergence of functions on the circle, it suffices to study the point $x=0$, and by translation invariance, we have the same conclusion hold for $x=x_0$.
\end{remark}

The convolution can be thought of as ``redistribution of mass,'' and we show the total mass of the Dirichlet kernel is 1.
\begin{equation*}
    \int_0^1D_N(t)dt=\int_0^1e^{2\pi ikt}dt=1
\end{equation*}

We now introduce a simple expression for the Dirichlet's kernel.
\begin{proposition}[Dirichelt kernel]
    We have
    \begin{equation*}
        D_N(t)=\frac{\sin(2N+1)\pi t}{\sin(\pi t)}
    \end{equation*}
\end{proposition}
\begin{proof}
    $D_N(t)=\sum_{k=-N}^Ne^{2\pi ikt}$ is a geometric series, with the ratio$=e^{-2\pi it}$, hence by the formula of partial sums, we have
    \begin{equation*}
        LHS=\frac{e^{2\pi i2Nt}(1-e^{-2\pi it(2N+1)})}{1-e^{-2\pi it}}
    \end{equation*}
    Now we examine the RHS. $\sin(t)=(e^{it}-e^{-it})/2i$, hence we have
    \begin{equation*}
        RHS=\frac{e^{(2N+1)\pi it}-e^{-(2N+1)\pi it}}{e^{i\pi t}-e^{-i\pi t}}
    \end{equation*}
    Dividing top and bottom by $e^{i\pi t}$ gives us the LHS.
\end{proof}
\qed

Can we comment on the bound of $D_N$? If one draws out a picture, then it is clear that $D_N$ have a blow-up at $t=0$, and $|D_N(t)|\to 2N+1$ as $t\to 0$, one could also see this using the expression above. Also we have the following:
\begin{equation*}
    |D_N(t)|\leq\frac{1}{\sin(\pi t)}, 0<t\leq\frac{1}{2}
\end{equation*}

To prove the above two theorems, we introduce some tools.
\begin{lemma}[Riemann-Lebesgue Lemma]
    If $f\in L^1(\mathbb{T})$, then as $|k|\to\infty$, the Fourier coefficent tends to 0, i.e. we have
    \begin{equation*}
        \lim_{|k|\to\infty}|\hat{f}(k)|=\lim_{|k|\to\infty}\left|\int_0^1f(x)e^{-2\pi ikx} \right|=0
    \end{equation*}
\end{lemma}
\begin{proof}
    For $f\in L^1(\T)$, there exists $g,h$ such that $f=g+h$, where $g$ is a simple function of the form $g=\sum_{i=1}^nc_i\chi_{E_i}$, and $\|h\|_{L^1}<\epsilon$.

    For $g$ simple function, the Fourier coefficent decays with rate $O(1/k)$ by integration by parts.
    \begin{equation*}
        |\hat{g}(k)|=\left|\int_0^1\sum_ic_i\chi_{E_i}e^{-2\pi ikx}\right|\leq\sum|c_i|\chi_{E_i}\left|\int_0^1e^{-2\pi ikx} \right|=\sum|c_i|\chi_{E_i}\frac{1}{2\pi ik}\left|\int_0^1xe^{-2\pi ikx} \right|\lesssim O(1/|k|)
    \end{equation*}

    For $\hat{h}(k)$, we have,
    \begin{equation*}
        \left|\hat{h}(k)\right|=\left|\int_0^1h(x)e^{-2\pi ikx}\right|\leq\int_0^1|h(x)|dx<\epsilon
    \end{equation*}
    Hence we have $|\hat{f}(k)|\leq O(1/k)$, and as $|k|\to\infty$, we have $\hat{f}(k)\to 0$ if $f\in L^1(\T)$.
\end{proof}
\qed

Next we a result that guarantees pointwise convergence.
\begin{theorem}[Riemann Localization principle]
    If $f=0$ in a neighborhood of $x$, say, $(x-\delta, x+\delta)$, then we have
    \begin{equation*}
        \lim_{N\to\infty}S_N(f)(x)=0=f(x)
    \end{equation*}
\end{theorem}
\begin{proof}
    If we write $S_N(f)$ as $D_N\ast f$, where $D_N(t)=\frac{\sin((2N+1)\pi t)}{\sin(\pi t)}$, then we see the badness of the denominator is avoided around $x$ by $f=0$ around $x$, and if we are far away from $x$, then nice bound ensues. We now do explicit computation.

    \begin{align*}
        S_N(f)&=\int_0^1 D_N(t)f(x-t)dt\\
        &\int_{\delta\leq t\leq 1}\frac{\sin((2N+1)\pi t)}{\sin(\pi t)}f(x-t)dt\\
        &=\int_{\delta\leq t\leq 1}\frac{f(x-t)}{\sin(\pi t)}\frac{e^{i(2N+1)\pi t}-e^{i(2N+1)\pi t}}{2i}dt\\
        &=\frac{1}{2i}\widehat{\frac{f(x-t)}{\sin(\pi t)}e^{i\pi t}}(-N)+\widehat{\frac{f(x-t)}{\sin(\pi t)}e^{-i\pi t}}(N)
    \end{align*}
    And we note that both $\frac{f(x-t)}{\sin(\pi t)}e^{i\pi t}$ and $\frac{f(x-t)}{\sin(\pi t)}e^{-i\pi t}$ are in $L^1(\T)$. Hence by the Riemann-Lebesgue lemma, their Fourier coefficients tend to 0 as $N\to\infty$. 
\end{proof}
\qed


\section{Lecture 3}
Recall Riemann localization principle.

\begin{remark}
    Given the localization principle, if we'd like to examine whether Fourier series converges at a particular point, we can assume $f=0$ outside a small neighborhood of that point.
\end{remark}

We now will prove Dini's criterion.

\begin{theorem}[Dini's Criterion]
    If $x\in\T$, $\delta>0$ such that
    \begin{equation*}
        \int_{|t|<\delta}\left|\frac{f(x+t)-f(x)}{t} \right|<\infty
    \end{equation*}
    Then the partial sum converges at $x$.
\end{theorem}
\begin{proof}
    Assume $\delta<\frac{1}{2}$, and we would like to show $S_N(f)(x)-f(x)\to 0$.
    \begin{align*}
        S_N(f(x)-f(x))&=\int_{|t|\leq1/2}(f(x-t)-f(x))D_N(t)dt\\
        &=\int_{|t|\leq\delta}(f(x-t)-f(x))D_N(t)dt+\int_{\delta\leq|t|\leq1/2}(f(x-t)-f(x))D_N(t)dt
    \end{align*}

    We notice that
    \begin{equation*}
        \left|\int_{|t|\leq\delta}(f(x-t)-f(x))D_N(t)dt\right|\leq\int_{|t|\leq\delta}\frac{|f(x-t)-f(x)|}{|\sin(t)|}<\infty
    \end{equation*}

    By the condition that the integral is $\infty$, we can choose a even smaller $\delta_1<\delta$ such that $\int_{|t|<\delta}<\epsilon$. Finishing the proof.
\end{proof}
\qed

\begin{remark}
     For Holder continuous functions, they all satisfy the the assumptions of Dini. Note, a function being continuosu does not guarantee that the partial sum of Fourier series would converge.
\end{remark}


Now we begin Jordan's Criterion.
\begin{theorem}[Jordan's Criterion]
    If $x\in\T$, and $f$ is of bounded variation in $(x-\delta, x+\delta)$, then we have
    \begin{equation*}
        \lim_{N\to\infty}S_Nf(x)=\frac{1}{2}(f(x+)+f(x-))
    \end{equation*}
\end{theorem}
\begin{remark}
    We don't expect $f$ to be continuous or even defined at $f(x)$.
\end{remark}
\begin{proof}
    WLOG, we assume $\delta<1/2$, and by the Riemann localization theorem, we assume $f=0$ outside of $(x-\delta, x+\delta)$. (What happens outside of a neighborhood of $\delta$ does not matter).

    Using the symmetry of the kernel. We have
    \begin{align*}
        S_Nf(x)(x)=\int_{|t|\leq1/2}f(x-t)D_N(t)dt=\int_0^{1/2}(f(x+t)+f(x-t))D_N(t)dt
    \end{align*}
    Let $g(t)=f(x+t)+f(x-t)$, and we would want to show that $\int_0^{1/2}g(t)D_N(t)dt\to\frac{1}{2}g(0)$.

    We can also assume $g$ is monotonic. And we want to show that $\int_{|t|\leq1/2}g(t)D_N(t)dt$ tends to $\frac{1}{2}g(0)$. Again we separate the integral into two parts and the second part is 0, and we use the second mean value theorem of definite integrals.
    \begin{lemma}
        For $f$ continuous and $g$ monotonic, we have
        \begin{equation*}
            \int_a^bfg=g(a+)\int_a^c f(t)dt+g(b-)\int_c^bf(t)dt
        \end{equation*}
    \end{lemma}
    We would want to show that $g(\delta-)\int_\nu^\delta f(t)dt$ tends to 0.

    \begin{align*}
        \left|\int_\nu^\delta D_N(t)dt\right|&=\left|\int_\nu^\delta \frac{\sin((2N+1)\pi t)}{\sin(\pi t)}\right|\\
        &\leq\left|\int\sin((2N+1)\pi t)(1/\sin(\pi t)-1/\sin(t))\right|+\left|\int\sin((2N+1)\pi t)\frac{1}{t} \right|\\
    \end{align*}
    The first term is bounded by a constant, and the second term, by integration by parts, is also bounded.

    \begin{remark}
        If we take $\delta_1$ arbitrarily small, then we get that the integral is arbitrarily small.
    \end{remark}
    Hence we can take $\delta$ to be arbitrarily small, to finish the proof.
\end{proof}
\qed




\section{Lecture 4}
We will now prove continuity is too rough of a condition to ensure Fourier series converges to $f$. Namely, we will show the existence of continuous functions whose Fourier series diverges at a particular point. 

\begin{theorem}
    There exists a continuous function whose ourier series diverges at a point.
\end{theorem}
\begin{proof}
    WLOG, we take this point to be $x=0$. We shall use UBP. Let $X=C(\T), Y=\C$, and equip $X$ with the $\|\cdot\|_\infty$ norm. Hence $X$ is a Banach space. 

    Then we define $T_N$ on $X$ as follows:
    \begin{equation*}
        T_N(f)=S_Nf(0)=\int_{|t|\leq 1/2}f(t)D(t)dt 
    \end{equation*}
    Hence it suffices to show $\|T_N\|=\infty$ as $N\to\infty$. We define a new quantity $L_N$ as follows:
    \begin{equation*}
        L_N=\int_{|t|\leq1/2}|D_N(t)|dt
    \end{equation*}
    We would like to show that $\|T_N\|=L_N$, and then prove $L_N$ tends to $\infty$ as $N\to\infty$.
    We first note that we have
    \begin{equation*}
        |T_N(f)|\leq\int |f(t)||D_N(t)|dt\leq\|f\|_\infty L_N
    \end{equation*}
    Then for the reverse direction, we note that $|D_N(t)|=D_N(t)\cdot sgn(D_N)(t)$, and by noting that $D_N(t)$ has only finitely many zeros, we can modify $sgn(D_N)$ into a function $f$ with $\|f\|_\infty$, and $f\in C(\T)$ and also that $|T_N(f)|\geq L_N-\epsilon$. Then we get
    \begin{equation*}
        \|T_N\|\geq L_N-\epsilon
    \end{equation*}
    Hence $\|T_N\|=L_N$. We next show, via computation, that $L_N\to\infty$ as $N\to\infty$.

    \begin{lemma}
        Define $L_N=\int_{-1/2}^{1/2}|D_N(t)|dt$, then we have
        \begin{equation*}
            L_N=\frac{4}{\pi^2}\log(N)+O(1)
        \end{equation*}
    \end{lemma}
    \begin{proof}
        \begin{align*}
            L_N&=2\int_0^{1/2}\left|\frac{\sin((2N+1)\pi t)}{\sin{\pi t}}\right|dt\\
            &=2\int_0^{1/2}\left|\frac{\sin((2N+1)\pi t)}{{\pi t}}\right|dt+O(1)\\
            &=2\int_0^{N+1/2}\left|\frac{\sin(\pi t)}{\pi t} \right|dt+O(1)\\
            &=\frac{2}{\pi}\sum_{k=0}^{N-1}\int_k^{k+1}\left|\frac{\sin(\pi t)}{\pi t} \right|dt + O(1)\\
            &=\frac{2}{\pi}\sum_{k=0}^{N-1}\int_{k}^{k+1}\left|\frac{\sin(\pi t)}{k+1} \right|dt + O(1)\\
            &=\frac{2}{\pi}\log(N)\cdot\frac{2}{\pi}+ O(1)
        \end{align*}
    \end{proof}
    \qed

    Now we've shown $\|T_N\|\to\infty$ as $N\to\infty$, then by UBP, there exists $f\in C(\T)$ such that $|T_N(f)|\to\infty$, the Fourier series diverges.
\end{proof}
\qed

Now we would like to address the next two questions on convergence in $L^p$ norm. Namely, do we necessarily have if $f\in L^p, 1\leq p<\infty$, then
\begin{enumerate}
    \item Do we have
     \begin{equation*}
        \lim_{N\to\infty}\|S_N(f)-f\|_{L^p}=0
    \end{equation*}
    \item Do we have
        \begin{equation*}
            \lim_{N\to\infty}S_N(f)(x)=f(x), \text{ for almost every } x
        \end{equation*}
\end{enumerate}

We will address Q1.
\begin{lemma}
    Let $1\leq p<\infty$, for all $f\in L^p$, we have $\lim_{N\to\infty}\|S_N(f)-f\|_{L^p}=0$ if and only if there exists a constant $C_p$ such that
    \begin{equation*}
        \|S_N(f)\|_{L^p}\leq C_p\|f\|_{L^p}
    \end{equation*}
\end{lemma}
\begin{proof}
    $(\Rightarrow)$ This follows from UBP. If $\|S_N\|$ is unbounded, then by UBP, there exists an $f$ such that $\|S_N(f)\|_{L^p}\to\infty$. Hence would not converge to $f$ in the $L^p$ sense.

    $(\Leftarrow)$ We note that trigonometric polynomials are dense in $L^p$, hence we can find $g$ a trig polynomial such that $\|f-g\|_{L^p}<\epsilon$. Hence we have,
    \begin{equation*}
        \|S_N(f)-f\|_{L^p}\leq\|S_N(f)-S_N(g)\|_{L^p}+\|S_N(g)-g\|_{L^p}+\|f-g\|_{L^p}
    \end{equation*}
    We note the first term is
    \begin{equation*}
        \|S_N(f-g)\|_{L^p}\lesssim \|f-g\|_{L^p}<\epsilon
    \end{equation*}
    Hence we get convergence in the $L^p$ norm.
\end{proof}
\qed

We note that when $1<p<\infty$, we indeed have
\begin{equation*}
    \|S_N(f)\|_{L^p}\leq C_p\|f\|_{L^p}
\end{equation*}
but for $p=1$, we have $\|S_N(f)\|_{L^1}=L_N$ as above, hence there is no convergence for $S_N(f)$ to $f$ in the $L^p$ sense. We will reformulate this theorem as follows.

\begin{theorem}[Convergence of $S_N(f)$ for $1<p<\infty$]
    If $1<p<infty$, for all $f\in L^p$, we have
    \begin{equation*}
        \lim_{N\to\infty}\|S_N(f)-f\|_{L^p}=0
    \end{equation*}
\end{theorem}

\section{Lecture 5}
We started late today.

Recall last night, we mentioned in order to ensure $S_N(f)$ converges to $f$ in the $L^p$ sense, we need to have uniform boundedness. specifically for $p=2$, the functions $e^{2\pi inx}$ form an orthonormal basis (by trig polynomials dense in $L^p$), hence we get
\begin{equation*}
    \|f\|_{L^2}^2=\sum_{n=-\infty}^\infty |\hat{f}(n)|^2
\end{equation*}

Last time we asked for $1\leq p<\infty$, and $f\in L^p$, whether we have
\begin{equation*}
    S_N(f)(x)\to f(x) \text{ almost everywhere }
\end{equation*}
We won't discuss in detail the answer to this question. For $p=1$, we know that this is not true, by Kolmogov constructing an integrable function whose Fourier series diverges almost everywhere. And we know that the answer is yes for all $p>1$.

We now begin section 1.5 of the book.
\begin{definition}[Cesaro sum]
    Define the $N$-th Cesaro sum of Fourier series as follows:
    \begin{equation*}
        \sigma_N(f)=\frac{1}{N+1}\sum_{k=0}^NS_k(f)
    \end{equation*}
\end{definition}
One can show, via the following,
\begin{align*}
    \sigma_N(f)(x)&=\frac{1}{N+1}\sum_[k=0]^NS_k(f)(x)\\
    &=\int_0^1f(t)\left(\frac{1}{N+1}\sum_{k=0}^ND_k(x-t) \right)dt\\
    &=\int_0^1f(t)F_N(x-t)dt
\end{align*}
where $F_n(t)=\frac{1}{N+1}\sum_{k=0}^ND_k(t)$, we call this the Fejer kernel, and one can show that 
\begin{equation*}
    F_N(t)=\frac{1}{N+1}\left(\frac{\sin((N+1)\pi t)}{\sin(\pi t)} \right)^2
\end{equation*}

We now note a few important and nice properties of $F_N(t)$. We have
\begin{enumerate}
    \item $F_N\geq 0$
    \item $\int_0^1 F_N(t)dt=\|F_N\|_{L^1}=1$
    \item $\lim_{N\to\infty}\int_{\delta\leq|t|\leq 1/2}F_N(t)dt=0$, for all $\delta>0$.
\end{enumerate}
Note that the third property follows from the fact that
\begin{equation*}
    F_N(t)\leq\frac{1}{(N+1)(\sin(\pi t)^2)}
\end{equation*}
Hence it's nicely behaved when staying away from 0.

\begin{theorem}
    For the following two cases,
    \begin{enumerate}
        \item $\forall f\in L^p$, where $1\leq p<\infty$
        \item $f\in C(\T)$, for $p=\infty$.
    \end{enumerate}
    we get convergence in $L^p$ norm for the Cesaro sum:
    \begin{equation*}
        \lim_{N\to\infty}\|\sigma_N(f)-f\|_{L^p}=0
    \end{equation*}
\end{theorem}
\begin{proof}[sketch]
    For $f\in L^p$, where $1\leq p<\infty$, we have 
    \begin{equation*}
        \int_{|t|<1/2}\|f(x-t)-f(x)\|_{L^p}F_N(t)dt
    \end{equation*}
    Hence separating into $\int_{|t|<\delta}$, and $\int_{\delta\leq|t|<1/2}$, the second one tends to 0, where the first one also tends to zero as we choose $\delta$ to be arbitrarily small and by $f\in L^p$. Hence the entire integral tends to 0. Similar proof for $f\in C(\T)$, and $p=\infty$.
\end{proof}
\qed

\section{Lecture 6}
From the above theorem, we are above to conclude the following statements regarding denseness of trig polynomials and the Fourier coefficients uniquely determine $f$, if $f$ is integrable.
\begin{corollary}
    For $1\leq p<\infty$, trigonometric polynomials are dense in $L^p$.
\end{corollary}
\begin{proof}
    Note that $\sigma_N(f)$ is a finite average of $S_k$'s, which are trigonometric polynomials.
\end{proof}

\begin{corollary}
    For $f\in L^1$, if $\widehat{f}(k)=0$, for all $k$, then $f$ is identically 0.
\end{corollary}

We now introduce a different summability method, by treating Fourier series as the formal limit on the unit circle.

\begin{comment}
    If you just try to make a text red, use \textcolor{color}{text}
\end{comment}

\textcolor{red}{Finish Poisson kernel!}

\section{Lecture 7}
We now define FT on the most intuitive class of functions.
\begin{definition}[FT on $L^1$]
    For $f\in L^1(\R^n)$, we define its Fourier transform to be the following transformation:
    \begin{equation*}
        \widehat{f}(\xi)=\int f(x)e^{-2\pi ix\cdot\xi}dx
    \end{equation*}
    where $x\cdot\xi=x_1\xi_1+...x_n\xi_n$. 
\end{definition}
FT has nice following properties that are easily verified.
\begin{proposition}
    \begin{enumerate}
        \item FT is linear. 
        \begin{equation*}
            \mathcal{F}(\alpha f+\beta g)=\alpha\hat{f}+\beta\hat{g}
        \end{equation*}
        \item Clearly $\|f\|_\infty\leq\|f\|_1$, and by DCT, $\hat{f}$ is continuous.
        \item Riemann-Lebesgue states that 
        \begin{equation*}
            \lim_{|\xi|\to\infty}\hat{f}(\xi)=0
        \end{equation*}
        \item Convolution in the physical space is pointwise multiplication in the Fourier space. $\widehat{f\ast g}=\hat{f}\hat{g}$
        \item Translation in the physical space is phase shift in the Fourier space; phase shift in the physical space is translation in the Fourier space.
        \begin{equation*}
            \widehat{f(x+h)}(\xi)=e^{2\pi ih\dot\xi}\hat{f}(\xi)
        \end{equation*}
        \begin{equation*}
            \widehat{e^{2\pi ih\cdot x}f}(\xi)=\hat{f}(\xi-h)
        \end{equation*}
        \item If $\rho$ is an orthogonal transformation, then 
        \begin{equation*}
            \widehat{f(\rho)}(\xi)=\hat{f}(\rho\xi)
        \end{equation*}
        \item FT under scaling. If $g(x)=\lambda^{-n}f(\lambda^{-1}x)$, then $\hat{g}(\xi)=\hat{f}(\lambda\xi)$
        \item FT with derivative. $\widehat{\partial_{x_j f}}=2\pi i\xi_j\hat{f}$.
        \item FT with multiplication by $x_j$. $\widehat{x_jf}=2\pi i\partial_{\xi_j}\hat{f}$.
    \end{enumerate}
\end{proposition}

Note that for finite dimensional measure spaces $\T$, by Holder's inequality $L^p(\T)$ embeds in $L^1(\T)$ for $p>1$. However, the same embedding does not hold for $L^p(\R^n)$, and $L^1(\R^n)$. We've only defined the Fourier transform on $L^1(\R^n)$, but not all other $p>1$ yet. To define FT on these spaces, we start by defining it on a nicer class, the Schwartz functions.

\begin{definition}[Schwartz Space]
    $f$ is in the Schwartz space, $\mathcal{S}(\R^n)$ if $f\in C^\infty$, and if $f$ and all its derivatives decrease rapidly at infinity. Rigorously, it means for all $\alpha, \beta\in\N^n$, we have
    \begin{equation*}
        \sup_x|x^\alpha D^\beta f(x)|<\infty
    \end{equation*}
\end{definition}

This means at infinity, the functions decay faster than any powers of $x$. Note that
\begin{equation*}
    \mathcal{D}\subset\mathcal{S}
\end{equation*}

However, there exists elements such as $e^{-|x|^2}$ the Gaussian, which do not have compact support, yet the Gaussian is a Schwartz function.


\begin{definition}[Tempered distributions]
    $\mathcal{S}'$ is the space of linear functions on $\mathcal{S}$. $T\in\mathcal{S}'$ if for every sequence $\{\phi_n\}\subset\mathcal{S}$, such that $\phi_n\to\phi$, then we have
    \begin{equation*}
        \lim_{n\to\infty}T(\phi_n)=T(\phi)
    \end{equation*}
\end{definition}
\begin{remark}
    A sequence of functions $\{\phi_n\}$ converges to $\phi\in\mathcal{S}$, if for all $\alpha, \beta$, we have
    \begin{equation*}
        \sup_x|x^\alpha D^\beta(\phi_n(x)-\phi(x))|<\epsilon
    \end{equation*}
\end{remark}

\begin{theorem}
    The Fourier transform on $\mathcal{S}$ is continuous, and $\mathcal{S}$ is closed under taking Fourier transforms.

    For any $f,g\in\mathcal{S}$, we have
    \begin{equation*}
        \int\hat{f}g=\int f\hat{g}
    \end{equation*}
    Moreover, for any $f\in\mathcal{S}$, we have the Fourier inversion formula:
    \begin{equation*}
        f(x)=\int_{\R^n} e^{2\pi ix\cdot\xi}f(\xi)d\xi
    \end{equation*}
\end{theorem}
\begin{proof}
    For $f,g\in\mathcal{S}$, we have $fg$ are integrable over $\R^n\times\R^n$. Hence
    \begin{equation*}
        \int\hat{f}g=\int f\hat{g}
    \end{equation*}
    follows from Fubini's theorem.
    In the following lemma, we show if $f=e^{-\pi|x|^2}$, then $\hat{f}=f$. Given this, we have that
    \begin{equation*}
        f(0)\int \hat{g}=g(0)\int\hat{f}
    \end{equation*}
    If we take $g(x)=e^{-\pi|x|^2}$, thus we have
    \begin{equation*}
        f(0)=\int\hat{f}(\xi)d\xi
    \end{equation*}
    Then a translation in the physical space is a phase shift in the Fourier space, hence we have
    \begin{equation*}
        f(x)=\int e^{2\pi ix\cdot\xi}\hat{f}(\xi)d\xi
    \end{equation*}
    \begin{lemma}[Gaussian is invariant under FT]
       If $f=e^{-\pi|x|^2}$, then we have
       \begin{equation*}
        \hat{f}(\xi)=e^{-\pi|\xi|^2}
       \end{equation*}
    \end{lemma}
    \begin{proof}
        We note that it suffices to show this for the $n=1$ case. And $f=^{-\pi x^2}$ solves the differential equation below:
        \begin{equation*}
            \begin{cases}
                u'+2\pi xu=0\\
                u(0)=1
            \end{cases}
        \end{equation*}
        To show $\hat{f}=f$, it suffices to show $\hat{f}$ solves the same equation and by uniqueness of solution.

        We take the Fourier transform of both sides, we get
        \begin{equation*}
            (u')^{\widehat{\phantom{.}}}+(2\pi xu)^{\widehat{\phantom{.}}}
        \end{equation*}
        Hence we get
        \begin{equation*}
            2\pi i\xi\hat{u}+\partial_\xi u=0
        \end{equation*}
        And also
        \begin{equation*}
            \hat{u}(0)=\int u(x)dx=\int e^{-\pi x^2}=1
        \end{equation*}
        Hence by uniqueness of ode, we get $\hat{u}=u$.
    \end{proof}
    \qed
\end{proof}
\qed

Once we have the Fourier inversion formula for $f\in\mathcal{S}$, we note that
\begin{equation*}
    (\hat{f})^{\widehat{\phantom{.}}}(x)=\int \hat{f}(\xi)e^{-2\pi ix\cdot\xi}d\xi=f(-x)
\end{equation*}
\begin{corollary}
    The Fourier transform has period 4.
\end{corollary}

Next, we define Fourier transform on the space of tempered distributions, $\mathcal{S}'$.
\begin{definition}
    If $T\in\mathcal{S}'$, then we define its Fourier transform as follows:
    \begin{equation*}
        \hat{T}(f)=T(\hat{f}), f\in\mathcal{S}
    \end{equation*}
\end{definition}
We do the check that $\hat{T}\in\mathcal{S}'$ is well defined. We check $\hat{T}$ is linear:
\begin{equation*}
    \hat{T}(f+g)=T(\hat{f}+\hat{g})=T(\hat{f})+T(\hat{g})=\hat{T}(f)+\hat{T}(g)
\end{equation*}
Similarly for scalar multiplication.

\begin{theorem}
    FT is a linear, bounded bijection from $\mathcal{S}'\to\mathcal{S}'$. Moreover, it is invertible and its inverse is bounded as well.
\end{theorem}
\begin{proof}
    It is bounded if and only if it is continuous, hence we check for $T_n\to T$ in $\mathcal{S}'$, we have, for any $f\in\mathcal{S}$,
    \begin{equation*}
        \hat{T_n}(f)=T_n(\hat{f})\to T(\hat{f})=\hat{T}(f)
    \end{equation*}
    Hence $\hat{T_n}\to\hat{T}$. And the invertability follows from FT has period 4. 
\end{proof}
\qed

\section{Lecture 8}
Now we've defined FT on $\mathcal{S}$, by density of $\mathcal{S}$ in $L^p$, we have a well-defined FT on $L^p$. We start by proving the Plancherel identity.
\begin{theorem}
    FT on $L^2$ is an isometry, i.e.
    \begin{equation*}
        \|f\|_{L^2}=\|\hat{f}\|_{L^2}, f\in L^2
    \end{equation*}
\end{theorem}
\begin{proof}
    For any $f,h\in\mathcal{S}$, we let $\hat{g}=\overline{h}$, then we have
    \begin{equation*}
        g=(\overline{h})^{\check{\phantom{.}}}=\int h(-x)e^{2\pi ix\cdot\xi}d\xi=\int h(y)e^{-2\pi iy\cdot\xi}d\xi=\hat{\overline{h}}
    \end{equation*}
    Hence by the duality formula, we have
    \begin{equation*}
        \int f\overline{h}=\int f\hat{g}=\int \hat{f}g=\int \hat{f}\hat{\overline{h}}
    \end{equation*}
    Setting $f=h$, we get
    \begin{equation*}
        \|f\|_{L^2}=\|\hat{f}\|_{L^2}
    \end{equation*}
\end{proof}
\qed

Next we stated Riesz-Thorin interpolation theorem and used it to show Hausdorff Young, and Young's inequality. We leave the proof of Riesz-Thorin as a homework question.
\begin{theorem}[Hausdorff Young]
    For $1\leq p\leq 2$, if $f\in L^p$, then $\hat{f}\in L^{p'}$, and 
    \begin{equation*}
        \|\hat{f}\|_{L^{p'}}\leq\|f\|_{L^p}
    \end{equation*}
\end{theorem}
\begin{proof}
    This follows from noting the inequality holds for $p=1, p'=\infty$, and $p=p'=2$.
\end{proof}
\qed

\begin{theorem}[Young's inequality]
    For $f\in L^p, g\in L^q$, if we let $\frac{1}{r}+1=\frac{1}{p}+\frac{1}{q}$, then we get
    \begin{equation*}
        \|f\ast g\|_{L^r}\leq\|f\|_{L^p}\|g\|_{L^q}
    \end{equation*}
\end{theorem}
\begin{proof}
    We note that 
    \begin{equation*}
        \|f\ast g\|_{L^p}=\left(\int\left|\int f(y)g(x-y)dy\right|^pdx\right)^{1/p}\leq\|f\|_{L^p}\|g\|_{L^1}
    \end{equation*}
    And by Holder, we have
    \begin{equation*}
        \|f\ast g\|_{L^\infty}\leq\|f\|_{L^p}\|g\|_{L^{p'}}
    \end{equation*}
    And then we apply Riesz-Thorin.
\end{proof}
\qed

\subsection{Convergence and partial summability methods}
Recall what we did with the Fourier series: given $S_N(f)=\sum_{n=-N}^N\hat{f}(n)e^{2\pi inx}$, what can we say about 
\begin{equation*}
    \lim_{N\to\infty}S_N(f)\to f(x)
\end{equation*}
And in both pointwise and $L^p$ sense. We now introduce the parallel concept with Fourier transform. We ask the question: when do we have
\begin{equation*}
    \lim_{R\to\infty}\int_{B_R}\hat{f}(\xi)e^{2\pi ix\cdot\xi}=f(x)
\end{equation*}
And we ask when this will converge to the original function.
\begin{definition}[Paritial Sum operator]
    Like how we want the partial sum of Fourier series to capture how $f$ behaves, we define the partial sum operator for Fourier transform as follows:
    \begin{equation*}
        (S_R(f))^{\widehat{\phantom{.}}}=\chi_{B_R}\hat{f}
    \end{equation*}
\end{definition}
Note that in the Fourier space, $S_R(f)\to\hat{f}$, and we would like to ask the question, does this $S_R(f)$ approximate $f$ in the physical space? In other words, this becomes the question whether we have
\begin{equation*}
    \lim_{R\to\infty}S_R(f)=f
\end{equation*}
Note that we have
\begin{equation*}
    S_R(f)=\check{X_{B_R}}\ast f:=D_R\ast f
\end{equation*}
where $D_R(x):=\int_{B_R}e^{2\pi ix\cdot\xi}d\xi$.

\begin{proposition}
    (Like with the Fourier series), we have
    \begin{equation*}
        \lim_{R\to\infty}\|S_R(f)-f\|_{L^p}=0
    \end{equation*}
    if and only if, there exists $C>0$ such that
    \begin{equation*}
        \|S_R(f)\|_{L^p}\leq C\|f\|_{L^p}
    \end{equation*}
\end{proposition}
\begin{remark}
    We have $\|S_R\|<\infty$ for $n=1$, but for $n>1$, this is only true for $p=2$. We will discuss this in the upcoming lectures.
\end{remark}
\begin{definition}[Dirichelt Kernel]
    We've defined the Dirichlet series as $D_N(x)=\sum_{n=-N}^Ne^{2\pi inx}$, and likewise, in $n=1$, we define the Dirichlet kernel as
    \begin{equation*}
        D_R(x)=\int_{-R}^R e^{2\pi ix\cdot\xi}d\xi
    \end{equation*}
\end{definition}
Note that
\begin{equation*}
    \int_{-R}^Re^{2\pi ix\cdot\xi}d\xi=\left.\frac{e^{2\pi ix\xi}}{2\pi ix}\right\vert_{-R}^R=\frac{1}{2\pi x}\sin(2\pi Rx)-\sin(-2\pi Rx)=\frac{\sin(2\pi Rx)}{\pi x}
\end{equation*}
\begin{lemma}
    (We might use this in the future). We have
    \begin{equation*}
        D_R\not\in L^1
    \end{equation*}
    However, for any $p>1$, we have
    \begin{equation*}
        D_R\in L^p
    \end{equation*}
\end{lemma}
\begin{proof}
    We have $\int_{-R}^R \sin(2\pi Rx)/(\pi x)$ not integrable, however for $p>1$, we have
    \begin{equation*}
        \int |D_R|^p\lesssim\int_{-R}^R \frac{1}{|x|^p}<\infty
    \end{equation*}
\end{proof}
\qed

\section{Lecture 9}
Recall we defined Dirichlet kernel for $n=1$, we first note that when $n>1$, we have $D_R$ integrates over a sphere with respect to a surface measure. Now we introduce partial summability methods just like with the Fourier series.
\begin{equation*}
    \sigma_R(f)=\frac{1}{R}\int_0^R S_r(f)dr
\end{equation*}
We have
\begin{equation*}
    \sigma_R(f)(x)=F_R\ast f(x)
\end{equation*}
where $F_R(x)=\frac{1}{R}\int_0^RD_r(x)dr$. 
We thus have
\begin{equation*}
    F_R(x)=\frac{1}{R\pi x}(1-cos(2\pi xR)/2\pi x)=\frac{\sin(\pi Rx)}{R(\pi x)^2}
\end{equation*}
We have
\begin{enumerate}
    \item $F_R\geq 0$
    \item $\|F_R\|_{L^1}=1$.
\end{enumerate}
Hence it's again, a redistribution of mass, and as $R\to\infty$, $F_R$ acts more and more like the dirac mass, the convolution identity. For $1\leq p<\infty$, we have
\begin{equation*}
    \lim_{R\to\infty}\|\sigma_R(f)-f\|_{L^p}=0
\end{equation*}

\subsection{Chapter 2}
We will talk about partial sum convergence, and better behaved summability methods for $S_Rf$, and we disucss both cases in the sense of $L^p$ and a.e. pointwise convergence.

\subsection{Chapter 2.1}
\begin{definition}[Approximation of the identity]
    Let $\phi\in L^1(\R^n)$, and $\int\phi=1$, and we define 
    \begin{equation*}
        \phi_t=t^{-n}\phi\left(\frac{x}{t} \right)
    \end{equation*} 
    And we call the family $\{\phi_t\}_{t>0}$ approxiation of the identity.
\end{definition}
We shall justify the name.
\begin{proposition}
    Let $\phi_t$ be defined above, then for any $f\in\mathcal{S}$, we have
    \begin{equation*}
        \lim_{t\to 0}\phi_t\ast f(x)=f(x)
    \end{equation*}
\end{proposition}
\begin{proof}
    It suffices to show that $\phi_t\to\delta$ in $\mathcal{S}'$, since it would imply
    \begin{equation*}
        \lim_{t\to 0}\phi_t\ast f(x)=\delta\ast f(x)=f(x)
    \end{equation*}
    And for any $f\in\mathcal{S}$, we have
    \begin{equation*}
        \phi_t(f)=\int t^{-n}\phi\left(\frac{x}{t}\right)f(x)dx=\int \phi(y)f(ty)dy
    \end{equation*}
    Hence as $t\to 0$, we have
    \begin{equation*}
        \lim_{t\to 0}\phi_t(f)=\int\phi(y)f(0)dy=f(0)
    \end{equation*}
    Hence $\phi_t\to\delta$, as $t\to 0$ in $\mathcal{S}'$.
\end{proof}
\qed

\begin{remark}
    Note that for Cesaro partial sum, we have the Fejer kernel as $F_R=\frac{1}{R}\int_0^R D_r(x)dr$, and in fact $F_R=\phi_{1/R}$.
\end{remark}
Next we see that the family of $\phi_t$ allows for convergence in $L^p$ norm as well.
\begin{theorem}
    Let $\{\phi_t\}_{t>0}$ be an approximation of the identity, then for any $f\in L^p$, $1\leq p<\infty$, we have
    \begin{equation*}
        \lim_{t\to 0}\|\phi_t\ast f-f\|_{L^p}=0
    \end{equation*}
\end{theorem}
\begin{proof}
    We have
    \begin{equation*}
        \phi_t\ast f=\int\phi(y)[f(x-ty)-f(x)]dy
    \end{equation*}
    Note that there exists $\delta>0$, for $|ty|<\delta$, we have
    \begin{equation*}
        \|f(x-ty)-f(x)\|_{L_x^p}<\frac{\epsilon}{2\|\phi\|_{L^1}}
    \end{equation*}
    Hence we separate the integral to $|y|<\frac{\delta}{t}$, and $|y|>\frac{\delta}{t}$. We thus have
    \begin{align*}
        \|\phi_t\ast f-f\|_{L^p}&\leq \int|\phi(y)|\|f(x-ty)-f(x)\|_{L_x^p}dy\\
        &=\int_{|y|<\delta/t}|\phi(y)|\|f(x-ty)-f(x)\|_{L_x^p}dy+\int_{|y|>\delta/t}|\phi(y)|\|f(x-ty)-f(x)\|_{L_x^p}dy\\
        &<\frac{\epsilon}{2\|\phi\|_{L^1}}\int_{|y|<\delta/t}|\phi(y)|dy+2\|f\|_{L^p}\int_{|y|>\delta/t}|\phi(y)|dy\\
        &<\epsilon
    \end{align*}
\end{proof}
\qed

This convergence in $L^p$ implies that there exists a sequence $t_n$ such that as $t_n\to 0$, for almost every $x$, we have
\begin{equation*}
    \lim\phi_{t_n}\ast f(x)\to f(x)
\end{equation*}


\section{Lecture 10}
Last time we introduced the approxiation to identity funcitons, and 
\begin{exercise}
    If $\{\phi_t\}$ are approxiation of identities and $f\ast \phi_t$ converges a.e. for all $f\in L^p, 1\leq p<\infty$, then it converges to $f$ pointwise almost everywhere.
\end{exercise}
\begin{remark}
    Note that you can pick $\{\phi_t\}$ compactly supported and smooth, and would result in a Fourier transform that preserves where $f$ originally lies, i.e. if $f\in L^p$, we have $\phi_t\ast f\in L^p$ as well. Even you can't take derivatives of $f$, $\phi\ast f$ is smooth for $\phi\in\mathcal{S}$, since convolution picks up the better derivatives.
\end{remark}

Now we go back to the question of almost everywhere pointwise convergence.

Fix $f$, we could approximate $f$ with a sequence by $\{f_n\}\subset\mathcal{S}$, for $f\in C_\infty(X)$, we do get almost everywhere convergence by remark from last time.

Now we look at
\begin{equation*}
    \phi_t(f-f_n)
\end{equation*}
we would want this as an ``error'' for all large $n$, and all small $t$ on some $E\subset\R^n$, such that $m(\R^n\setminus E)<\epsilon$. This guarantees almost everywhere convergence, $\phi_t(f_n-f)$ could behave wildly on $\R^n\setminus E$, and it would not have mattered.

Note this is a two parameter family in $t$ and $n$, taking advantage of what $\phi_t$ have in common, for all $t$.

\begin{enumerate}
    \item $\int \phi_t=1$.
    \item They all seem to have some sort of bump near 0, and could be thought of as an average.
\end{enumerate}

\subsection{Weal type inequalities and a.e. convergence}


\begin{theorem}
    Let $\{T_t\}$ be a family linear of sublinear operators on $L^p(X,\mu)$, $t\in A\subset\R$, and define
    \begin{equation*}
        T^*f(x)=\sup_t|T_tf(x)|
    \end{equation*}
    this is defined as the maximal function. If $T^*$ is of weak type-$(p,q)$, then the set
    \begin{equation*}
        \{f\in L^p(X,\mu): \lim_{t\to t_0}T_t(f)(x)=f(x), a.e. \}
    \end{equation*}
    is closed in $L^p(X,\mu)$.
\end{theorem}
\begin{remark}
    If $T: L^p\to L^q$, then if $T$ is bounded, we call it strong-type $(p,q)$, and strong type implies that $T$ is of weak type $(p,q)$. In most situations, we will define $p=q$.
\end{remark}

Let $(X,\mu)$, $(Y,\nu)$ be measure spaces, and $T$ be an operator on $L^p(X,\mu)$ to measurable functions on $L^q$. All elements are $\geq 1$.
\begin{definition}[weakly bounded]
    For $q<\infty$, $T$ is weak type $(p,q)$ if there exists $C>0$,
    \begin{equation*}
        \nu\left(\{y\in Y: |T(f)(y)|>\lambda\} \right)\leq\left(\frac{C\|f\|_{L^p}}{\lambda} \right)^q
    \end{equation*}
\end{definition}
\begin{example}
    For $p=q=1$, we take $X=Y=\R$. We allow $T(f)$ to be the function $1/x$, which is not bounded of strong type.
\end{example}

\begin{exercise}
    Weak type $(p,\infty)$ is equivalent to saying bounded from $L^p$ to $L^\infty$.
\end{exercise}
\begin{proof}
    For $T$ of weak type $(p,\infty)$, we have
    \begin{equation*}
        \lambda\left(\nu\left(\{y\in Y: |T(f)(y)|>\lambda\} \right) \right)^\frac{1}{p}\leq C\|f\|_{L^p}
    \end{equation*}
    Hence we have $p\to\infty$, $(\nu\left(\{y\in Y: |T(f)(y)|>\lambda\} \right))^\frac{1}{p}\to 1$ for any $\nu\left(\{y\in Y: |T(f)(y)|>\lambda\} \right)\neq 0$, hence if we let $\lambda\to\infty$, we get a contradiction. Hence we could allow allow $\nu\left(\{y\in Y: |T(f)(y)|>\lambda\} \right)<\epsilon$, for all $\epsilon>0$.
\end{proof}
\qed

\begin{exercise}
Strong type $(p,q)$ implies the weak type $(p,q)$.
\end{exercise}
\begin{proof}
    We denote the set $\{y\in Y: |T(f)(y)|>\lambda\}$ as $E_\lambda$, then we have
    \begin{equation*}
        \nu(E_\lambda)=\int_{E_\lambda}1d\nu\leq\int_{E_\lambda}\left(\frac{|T(f)y|}{\lambda}\right)^qd\nu\leq\left(\frac{C\|f\|_{L^p}}{\lambda}\right)^q
    \end{equation*}
    in the last inequality, we used the fact that $T$ is bounded from $L^p$ to $L^q$.
\end{proof}
\qed

\begin{proof}[of theorem]
    We would like to show the set is closed, hence we take a sequence of $\{f_n\}$ such that $\|f_n-f\|_{L^p}\to 0$, and $T_t(f_n)(x)\to f_n(x)$ for almost every $x$. 

    WLOG assume $t_0=0$. Assume $f_n\to f$ in $L^p$, and that
    \begin{equation*}
        T_t(f_n)(x)\to f_n(x) \text{ for almost every } x
    \end{equation*}
    as $t\to 0$ a.e. for every $f_n$.

    The idea of ``oscillation.'' The goal is to show $T_t(f)\to f$ as $t\to 0$ almost everywhere.

    Define
    \begin{equation*}
        W_\lambda=\{x: \limsup_{t\to 0}|(T_t(f)-f)(x)|>\lambda \}
    \end{equation*}
    for each $\lambda>0$. We are done if we show 
    \begin{equation*}
        \mu(W_\lambda)=0
    \end{equation*}
    Note that 
    \begin{equation*}
        W_\lambda\subset\{x:\limsup_{t\to 0}|(T_t(f)-T_t(f_n))(x)|>\lambda/3\}\cup \{x: |(f-f_n)(x)|>\lambda/3\}
    \end{equation*}
    for any fixed $n$. Note that if we denote the first set as $A$, the second as $B$, then 
    \begin{equation*}
        \mu(B)\lesssim \left(\frac{\|f-f_n\|_{L^p}}{\lambda} \right)^p
    \end{equation*}
    For $A$, note that 
    \begin{equation*}
        A\subset \{x: T^*(f-f_n)(x)\geq\lambda/3\}
    \end{equation*}
    By weak type $(p,q)$ of $T^*$, hence we have
    \begin{equation*}
        \mu(B)\lesssim \left(\frac{\|f-f_n\|^p}{\lambda} \right)^q
    \end{equation*}
    We can make $\mu(A), \mu(B)<\epsilon$, by choosing $n$ large enough. Hence we have
    \begin{equation*}
        \mu(W_\lambda)<2\epsilon
    \end{equation*}
\end{proof}
\qed

\begin{definition}[Sublinear operator]
    $T$ is called sublinear, we have 
    \begin{equation*}
        T(f+g)\leq T(f)+T(g), T(\lambda f)=\lambda T(f)
    \end{equation*}
\end{definition}

\section{Lecture 11}
We will begin with one comment. (This is in 8.3 of the book)
\begin{definition}[Lorentz space]
    For $1\leq p<\infty$, we take $f\in L^{p,\infty}$ if and only if
    \begin{equation*}
        \lambda^p|\{|f|\geq\lambda\}|<C(f), \text{ for all } \lambda
    \end{equation*}
\end{definition}
Note that this connects to the weak type definition that we introduced,
we have $T$ is weak $(p,p)$, if and only if 
\begin{equation*}
    \|Tf\|_{p,\infty}\lesssim \|f\|_{L^p}
\end{equation*}
Note that $\|\cdot\|_{p,\infty}$ is a quasinorm, however, for $p\neq 1$, we have $\|\cdot\|_{p,\infty}$ generates a normable sapce.

\begin{definition}[dyadic number]
    For $k\in\mathbb{Z}$, we have $\lambda=2^k$ is a dyadic number.
\end{definition}
We could decompose $\R$ into dyadic intervals.
\begin{equation*}
    \R^+=...\bigsqcup (1,2]\bigsqcup (2,4]...
\end{equation*}
This type of decomposition is mild, i.e. if we try to estimate the $L^p$ norm of a function, then it is not too different from knowing how $f$ behaves within one $\lambda$ interval.

\begin{definition}[mid-level sets]
    For dyadic $\lambda$, define
    \begin{equation*}
        b_\lambda(f)=|\{\lambda: \lambda<|f(x)|\leq 2\lambda\}|
    \end{equation*}
\end{definition}
\begin{exercise}
    \begin{enumerate}
        \item $f\in L^p$ if and only if $\sum_{\lambda, dyadic}\lambda^pb_\lambda(f)<\infty$.
        \item $f\in L^p$, if and oly if each $\lambda^pb_\lambda(f)$ is uniformly bounded.
    \end{enumerate}
\end{exercise}
\begin{remark}
    Note that the only way for $\sum_\lambda^p b_\lambda(f)$ to blow up is to have many $b_\lambda(f)$ to be 1 for n many lambda's, i.e. blow up in a log fashion.
\end{remark}
We recall that for $f\in\mathcal{S}$, we have
\begin{equation*}
    \phi_t\ast f\to f \text{ pointwise a.e.}
\end{equation*}

Hence if $\limsup_{t\to 0}|\phi_t\ast f(x)|$ is weak type $(p,p)$, we know a.e. convergence for $p\in [1,\infty)$. This is indeed true, and we will prove it in the upcoming lectures.
\begin{exercise}
    Show that this holds for weak$ (\infty,\infty)$.
\end{exercise}

\section{Chapter 2.3 Marcinkiewicz interpolation theorem}
This interpolation thoerem allows us to reduce the case to $p=1$ (given that we know it holds for $p=\infty$). for the convergence of $|\phi_t\ast f|$.

\begin{remark}
    We cannot apply Riesz Thorin, since $\limsup_{t\to 0}|\phi_t\ast f(x)|$ is not linear.
\end{remark}

\begin{theorem}[Marcinkiewicz interpolation theorem]
    Let $(X,\mu), (Y,\nu)$ be measure spaces, and $1\leq p_0<p_1\leq\infty$. Let $T$ be a sublinear operator form $L^p(X,\mu)+ L^{p_1}(X,\nu)$ to measurable functions on $Y$.

    If $T$ is weak type $(p_0, p_0)$ and weak $(p_1, p_1)$, then $T$ is strong type $(p,p)$, for $p_0<p<p_1$.
\end{theorem}
\begin{remark}
    Marcinkiewicz is applicable to sublinear operators, but only from $L^p\to L^p$. Riesz-Thorin only allows for linear operators, but could go from $L^p\to L^q$. The complex interpolation that stems from Riesz-Thorin also can be applied to a family of operators.
\end{remark}
\begin{remark}
    Tomas-Stein constructed a family of operators, and then we applied Riesz-Thorin.
\end{remark}
We first introduce the definition of distribution function of $f$.
\begin{definition}[distribution function of $f$]
    Let $f:X\to\C$ be measurable, define 
    \begin{equation*}
        a_f(\lambda)=\mu(\{x\in X: |f(x)|>\lambda\})
    \end{equation*}
    to be the distribution function of $f$, and note that it is a decreasing function of $\lambda$. Sets of the form $\{x\in X:|f(x)|>\lambda\}$ are sometimes called the super-level sets.
\end{definition}


\begin{proposition}
    Let $\phi:[0,\infty)\to [0,\infty)$ be differentiable, increasing with $\phi(0)=0$, then we have
    \begin{equation*}
        \int_X\phi(|f(x))|d\mu=\int_0^\infty\phi'(\lambda)a_f(\lambda) d\lambda
    \end{equation*}
\end{proposition}
\begin{proof}
    We note that for the LHS, we have
    \begin{equation*}
        \int_X\phi(|f(x)|)d\mu=\int_X\int_0^{|f(x)|}\phi'(\lambda)d\lambda d\mu=\int_0^\infty\phi'(\lambda)\int_{\{x:|f(x)|>\lambda\}}d\mu d\lambda=\int_0^\infty\phi'(\lambda)a_f(\lambda)d\lambda
    \end{equation*}
\end{proof}
\qed

\begin{corollary}[$L^p$ norm in terms of $a_f(\lambda)$]
    Following is an alternative way to express the $L^p$ norm in terms of the distribution function of $f$.
    \begin{equation*}
        \int|f(x)|^pdx=\int_0^\infty p\lambda^{p-1}a_{f}(\lambda)d\lambda
    \end{equation*}
\end{corollary}
\begin{exercise}
    What is the LHS in terms of $b_f(\lambda)$.
\end{exercise}
\begin{proof}[of the interpolation theorem]

Let $f=f_0+f_1$, where $f_0=f\chi_{x:|f(x)|>c\lambda}$, and $f_1=f\chi_{x:|f(x)|\leq c\lambda}$, for some $c>0$ to be determined later.

We have a simple observation that
\begin{equation*}
    a_{Tf}(\lambda)\leq a_{Tf_0}\left(\frac{\lambda}{2} \right)+a_{Tf_1}\left(\frac{\lambda}{2} \right)
\end{equation*}
We will prove the $p_1<\infty$ case.
\begin{exercise}
    Prove it for $p_1=\infty$ case.
\end{exercise}
\begin{proof}
    For $p_1=\infty$, we have $T$ as a bounded operator from $L^\infty$ to $L^\infty$, hence we denote $\|T\|_{L^\infty\to L^\infty}=A_1$. 


\end{proof}


Note that we have
\begin{align*}
    \int_X|Tf|^pd\nu&=\int_0^\infty p\lambda^{p-1}a_{Tf}(\lambda)d\lambda\\
    &\leq p\int_0^\infty\lambda^{p-1}a_{Tf_0}\left(\frac{\lambda}{2} \right)d\lambda+p\int_0^\infty\lambda^{p-1}a_{Tf_1}\left(\frac{\lambda}{2} \right)d\lambda\\
    &=p\int_0^\infty\lambda^{p-1}\left(\frac{A_0}{\lambda/2}\|f_0\|_{L^{p_0}} \right)^{p_0}d\nu+p\int_0^\infty\lambda^{p-1}\left(\frac{A_1}{\lambda/2}\|f_1\|_{L^{p_1}} \right)^{p_1}d\nu\\
    &\leq p\int_0^\infty \lambda^{p-1}A_0^{p_0}2^{p_0}\lambda^{-p_0}\int_{\{x: |f(x)|>c\lambda\}}|f(x)|^{p_0}d\mu+\text{ similar term }\\
    &=p\int |f(x)|^{p_0}2^{p_0}A_0^{p_0}\int_0^{|f(x)|/c}\lambda^{p-p_0-1}d\lambda
\end{align*}
Optimizing $c$, we have
\begin{equation*}
    \|Tf\|_{L^p}2p^{1/p}\left(\frac{1}{p-p_0}+\frac{1}{p_1-p} \right)^\frac{1}{p}A_0^{1-\theta}A_1^\theta\|f\|_{L^p}
\end{equation*}

where $A_0=\|T\|_{L^{p_0}\to L^{p_0}}, A_1=\|T\|_{L^{p_1}\to L^{p_1}}$.


Note that for the first term, we have
\begin{equation*}
    \|f_0\|_{p_0}^{p_0}=\int|f_0|^{p_0}d\mu\leq (c\lambda)^{p_0-p}\int|f_0|^pd\mu=(c\lambda)^{p_0-p}\|f\|_{L^p}^p
\end{equation*}

\end{proof}
\qed



\section{Lecture 12}
We now begin our discussion of Hardy-Littlewood maximal function.


\begin{definition}[Hardy-Littlewood maximal function]
    Let $f\in L_{loc}^1(\R^n)$, then we define the Hardy-Littlewood maximal function as followS:
    \begin{equation*}
        Mf(x)=\frac{1}{|B_r|}\int_{B_r}|f(x-y)|dy
    \end{equation*}
    where $B_r$ is the ball of radius of $r$ around 0. 
\end{definition}
Note that this is equivalent to $M'f$ where the balls are replaced with cubes, and we could non-center the tubes to define $M''(f)$.

\begin{theorem}
    $M$ is weak (1,1) and strong $(p,p)$ for $1<p\leq\infty$.
\end{theorem}
\begin{proof}
    We have the following:
    \begin{equation*}
        \|Mf\|_\infty\leq\|f\|_{\infty}
    \end{equation*}
    Hence $M$ is bounded from $M:L^\infty\to L^\infty$.
    
    Hence by Marcinkiewicz interpolation theorem, it suffices to prove that $M$ is weak (1,1). We need to investigate $a_{Mf}$. 
    \begin{lemma}
        Let $\{I_\alpha\}$ be a collection of open intervals of $\R$, and let $K$ be a compact subset such that $K\subset\bigcup_\alpha I_\alpha$. There exists a finite subcover such that $\{I_j\}_{j=1}^n$ such that for each $x\in \R$, it is not contained in more than two $I_j$'s.
    \end{lemma}
    \begin{exercise}
        complete the proof of the lemma. Hint: for each $x$, consider all the intervals that $x$ is contained in, and only keep the left and right most ones.
    \end{exercise}

    Now for $n=1$, we let
    \begin{equation*}
        E_\lambda=\{x\in\R: Mf(x)>\lambda\}
    \end{equation*}
    We would like to show that $\mu(E_\lambda)\lesssim \|f\|_{L^1}$. It suffices to show this for any $K$ compact contained in $E_\lambda$.

    Note that for each $x\in E_\lambda$, there exists an interval $I_x$ such that 
    \begin{equation*}
        \int_{I_x}|f|\leq\lambda|I_x|
    \end{equation*}
    Hence by lemma above, for any compact set $K$ contained in $I_x$, there exists a finite subcollection of $\{I_j\}$, such that 
    \begin{equation*}
        \sum_j\chi_{I_j}\leq 2
    \end{equation*}
    Thus we have the following:
    \begin{equation*}
        \lambda|K|\leq\lambda\sum_j|I_j|\leq\sum_j\int_{I_j}|f|dy= \int_\R|f(y)|\sum_j\chi_{I_j}dy\leq \lambda2\|f\|_{L^1}
    \end{equation*}
\end{proof}
\qed

\begin{proposition}
    Let $\phi$ be positive, radial and decreasing, and $\phi\in L^1$, then 
    \begin{equation*}
        \sup_{t>0}|\phi_t\ast f(x)|\leq\|\phi\|_{L^1}Mf(x)
    \end{equation*}
\end{proposition}
\begin{proof}
    One could show this using approximation, i.e. approximate $\phi$ by step funcitons such that in Figure 1. We let $\phi=\sum_ja_j\chi_{B_j}(x)$, with $a_j\geq 0$. It suffices to show the proposition holds for $\phi$ simples functions like above.

    We have
    \begin{equation*}
        \sup_{t>0}\|\phi_t\ast f(x)\|\leq\sum_j\sup_{t>0}|a_j\chi_{B_j}\ast f(x)|\leq\sum_j a_j|B_j|\frac{1}{|B_j|}\int_{B_j}f(x-y)\leq \sum_ja_j|B_j|Mf(x)=\|\phi\|_{L^1}Mf(x)
    \end{equation*}
    Then we approximate $\phi$ using this in the general case.
\end{proof}
\qed

\begin{corollary}
    If $|\phi|\leq\psi$, $\psi$ such that it is positive, radial, and decreasing, then we have
    $\sup_t|\phi_t\ast f|$ is weak (1,1), and strong $(p,p)$, for $1<p\leq\infty$.
\end{corollary}
Hence we resolved a pointwise convergence issue that we raised previously.
\begin{corollary}
    If $|\phi|\leq\psi$, and $\psi$ is such that is positive, radial and decreasing, then in particular for Cesaro, Abel-Poisson, and Guass-Weierstrass kernels, we have if $f\in L^p$, $1\leq p<\infty$, or if $f\in C_0$, then we have
    \begin{equation*}
        \lim_{t\to 0}\phi_t\ast f=\left(\int\phi(y)dy\right)\cdot f, \text{ almost everywhere }
    \end{equation*}
\end{corollary}


\subsection{Dyadic maximal function}
We define a $r$-cube in $\R^n$ by $[0,1)^n$, and translating it.

\begin{definition}
    We define $Q_0=\{\text{ lattice cubes in } \R^n\}$, where they are unit cubes such that the corners are lattices. We define the dyadic lattice cubes as follows:
    \begin{equation*}
        Q_k=\{ 2^{-k}-\text{ unit cubes in} \R^n, k\in\mathbb{Z}\}
    \end{equation*}
    $\bigcup_{k\in\mathbb{Z}}Q_k$ is dyadic cubes.
\end{definition}


