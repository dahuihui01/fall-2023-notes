\chapter{Introduction}
We will first introduce three questions in incidence geometry: the projection problem, the distance set problem, and the discrete Kakeya problem in $\R^2$. Let $P$ be a discrete subset of $\R^2$.

\begin{problem}[ (Projection)]
    Let $e\in S^1$, and $\pi_e$ be the projection onto the line $l_e$. We ask the upper bound on the number of $e$ such that $\pi_e(P)\leq\frac{n}{8}$, given that $P$ is a discrete set with $|P|=n$. 
\end{problem}
\begin{problem}[ (Distance set)]
    What is the lower bound the distance set $\Delta(P)$
    \begin{equation*}
        \Delta(P)=\{|p-p'|: p,p'\in P\}
    \end{equation*}
\end{problem}
\begin{problem}[ (Discrete Kakeya/Joints problem)]
    Given a set of $m$ lines $\mathcal{L}$, such that each line $l\in\mathcal{L}$ is $m$-rich, i.e.
    \begin{equation*}
        |P\cap l|\geq m \text{ for each } l
    \end{equation*}
    Can we put a lower bound on the size of $P$.
\end{problem}

We remind ourselves of a sharp bound regarding how the lines and points intersect. Let $I(P,\lin)=\{(p,l)\in P\times \lin: p\in l\}$ 
\begin{theorem}[Szemeredi-Trotter theorem]
    For any $P\subset\R^2$, and a finite set of lines, then we have
    \begin{equation*}
        |I(P,\lin)|\lesssim \left(|P||\lin|\right)^\frac{2}{3}+|\lin|+|P|
    \end{equation*}
\end{theorem}
We will prove a weaker result for some intuition, and gain some insight into the projection problem and the discrete Kakeya problem.
\begin{proposition}[Weaker S-T]
    \label{weakst}
    In $\R^2$, we have that
    \begin{equation}
        |I(P,\lin)|\lesssim 4\min\{|P|^\frac{1}{2}|\lin|+|P|, |\lin|^\frac{1}{2}|P|+|\lin|\}
    \end{equation}
\end{proposition}
Using Proposition~\ref{weakst}, we get the following lower bound on the discrete Kakeya problem in $\R^2$.
\begin{corollary}
    we get that for a set of $m$ lines such that each line intersects the point set $P$ at least $m$ times, we get that
    \begin{equation*}
        |P|\gtrsim m^2
    \end{equation*}
\end{corollary}
\begin{note}
    The distance set problem can be realized as intersections between points and circles, instead of points and lines.
\end{note}
We make a similar conjecture in $\R^n$, for $m^{n-1}$ lines such that each line intersects the point set $P$ at least $m$ times, then we should have
\begin{equation*}
    |P|\gtrsim m^n
\end{equation*}
This statement fails for $\R^3$. Yet we could enforce some assumption to push to a nicer result.
\begin{theorem}[G-N, Joints Problem]
    For a set of $m^2$ lines such that no more than $m$ lines lie in the same plane, and each line intersects the point set $P$ at at least $m$ points, then we have
    \begin{equation*}
        |P|\gtrsim m^3
    \end{equation*}
    (This is in fact a conjecture by Bourgain and a corollary to the Joints problem in $\R^3$).
\end{theorem}
We now prove Proposition~\ref{weakst}. \textcolor{red}{unfinished here, the key idea is to use cauchy schwartz to get an $l^2$ norm to interpret as two points. } 

We now give some general bounds on the size of $\Delta(P)$ given that $|P|=n$.
\begin{exercise}
    For a given $n\in\mathbb{N}$, there exists a set $P$ such that $|\Delta(P)|\lesssim n$, for example, the set of $n$ points arranged on a straight line.
\end{exercise}
\begin{exercise}
    We now get some general lower bound on $\Delta(P)$. We can show $|\Delta(P)|\gtrsim n^\frac{1}{2}$. Consider two distinct points $p_1, p_2$, if we show that either
    \begin{equation*}
        |\{|p_1-p|:q\in P\}|\gtrsim n^\frac{1}{2} \text{ or } |\{|p_2-q|: q\in P\}|\gtrsim n^\frac{1}{2}
    \end{equation*}
    WLOG, assume $p_1$ has that
    \begin{equation}\label{circle}
        |\{|p_1-q|: q\in P\}|\lesssim n^\frac{1}{2}
    \end{equation}
    Then we would like to show that
    \begin{equation*}
        |\{|p_2-q|: q\in P\}|\gtrsim n^\frac{1}{2}
    \end{equation*}
    If the equation~\ref{circle} is true, then there exists a distance $r$ such that 
    \begin{equation*}
        |Q|=|\{q\in P:|p_1-q|=r|\}|\gtrsim n^\frac{1}{2}
    \end{equation*}
    And for $p_1\neq p_2$, we have
    \begin{equation*}
        |\{|p_2-q|:q\in Q\}|\gtrsim n^\frac{1}{2}
    \end{equation*}
\end{exercise}

\chapter{Dimensions}
We now discuss some ways of measuring size of fractal sets.
\begin{definition}
    Given a bounded set $E$, we define its $\delta$-covering number $|E|_\delta$ as the smallest number of $\delta$-balls needed to cover $E$.
\end{definition}
We note that as $\delta\to 0$, $|E|_\delta\to\infty$, so does $\frac{1}{\delta}$, hence comparing the rate of increase between the two gives us the Minkowski dimension (box counting dimension).
\begin{example}
    Let $f: (X,d)\to (Y, d')$ is biLipschitz, if there eixts a constant $C$ such that 
    \begin{equation*}
        C^{-1}d'(f(x), f(y))\leq d(x,y)\leq Cd'(f(x), f(y))
    \end{equation*}
    Let $f:[0,1]^n\to\R^n$ be biLipschitz, where $E=f([0,1]^n)$, then we have
    \begin{equation*}
        C^{-1}E\leq |[0,1]^n|\leq CE
    \end{equation*}
    Hence $[0,1]\sim E$, and $|E|_\delta\sim\delta^{-n}$.
\end{example}

\begin{definition}[Upper and Lower Minkowski's dimension]
    Let $E$ be a bounded set in $\R^n$, and $|E|_\delta$ be the $\delta$-covering number, then we define the upper and lower Minkowski dimension as follows:
    \begin{equation*}
        \overline{\dim_B}(E)=\limsup_{\delta\to 0}\frac{\log(|E|_\delta)}{\log(1/\delta)}, \underline{\dim_B}(E)=\liminf_{\delta\to 0}\frac{\log(|E|_\delta)}{\log(1/\delta)}
    \end{equation*}

\end{definition}
\begin{example}\label{exampleQ}
    The countable set $E=\mathbb{Q}
    \cap[0,1]$, has Lebesgue measure 0, and has Minkowski dimension:
    \begin{equation*}
        \dim_B(E)=\lim_{\delta\to 0}\frac{\log(\delta^{-1})}{\log(\delta^{-1})}=1
    \end{equation*}
\end{example}
\begin{example}
    The set $E=\{\frac{1}{n}:n\in\mathbb{N}\}$ has Minkowski dimension: for every $\frac{1}{n}$, it could be covered by a $\delta=n^{-2}$-length disjoint interval, hence
    \begin{equation*}
        \dim_B(E)=\lim_{\delta\to 0}\frac{\log(n)}{\log(n^2)}=\frac{1}{2}
    \end{equation*}
\end{example}
\begin{example}
    The set $E=\{\frac{1}{2^n}: n\in\mathbb{N}\}$ is ``too sparse'' of a fractal so its box counting dimension is the same as the topological dimension.
    \begin{equation*}
        \dim_B(E)=\lim_{\delta\to 0}\frac{\log(n)}{\log(2^n)}=\lim_{n\to\infty}\frac{\log(n)}{n\log(2)}=0
    \end{equation*}
    One could generalize this to get any set $E=\{a^{-n}: n\in\mathbb{N}\}$ has Minkowski dimension 0.
\end{example}
\begin{example}
    The Cantor set, splits into $2^n$ intervals of length $\frac{1}{3^n}$.
    \begin{equation*}
        \dim_B(E)=\lim_{\delta\to0}\frac{\log(2^n)}{\log(3^n)}=\frac{\log(2)}{\log(3)}
    \end{equation*}
\end{example}
\begin{note}
    Minkowski dimension does not always exist if the upper or lower Minkowski dimensions don't agree, and it does not work with unbounded sets $E$.
\end{note}
\begin{note}
    The example~\ref{exampleQ} has Minkowski dimension 1, but it is a countable set, hence we would like to assign it measure 0.
    \begin{equation*}
        \dim\cup_iE_i=\sup_{i}\dim E_i
    \end{equation*}
\end{note}
To address the above two concerns, we introduce the Hausdorff dimension. We do it in three steps: introduce an up-to-$\delta$-cover $\{U_j\}$, construct Hausdorff $\delta$-measure, and letting $\delta\to 0$.
\subsection{Hausdorff measure}
\begin{definition}[$s$-dim Hausdorff measure]
    Fix $s\geq 0$, and $\delta\in(0,\infty]$, given a set $E\in\R^n$, an ``up-to-$\delta$''-cover of $E$ is a \textbf{countable} family of sets $\{U_j\}_{j\in\mathbb{N}}$ such that
    \begin{equation*}
        E\subset\cup_jU_j, diam(U_j)\leq\delta, \text{ for all } j
    \end{equation*}
    And an $s$-dimensional Hausdorff $\delta$-meausre of the set $E$ is
    \begin{equation*}
        H_\delta^s(E)=\inf\left\{\sum_jdiam(U_j)^s, \{U_j\}_j \text{ is an up-to-$\delta$-cover of } E \right\}
    \end{equation*}
    Finally, the $s$-dimensional Hausdorff measure of $E$ is
    \begin{equation*}
        H^s(E)=\lim_{\delta\to 0}H_\delta^s(E)
    \end{equation*}
\end{definition}
\begin{remark}
    The limit is well justified since as $\delta\to 0$, $H_\delta^s(E)$ is an increasing function.
\end{remark}
There are many nice properties regarding the Hausdorff measure, for example, $n$-dim Hausdorff measure agrees with the $n$-dim Lebesgue measure, and there is a unique number such that the Hausdorff measure stops being $\infty$, and equivalently drops to zero. Hence based on this observation, we introduce the Hausdorff dimension of a set $E$.
\begin{definition}[Hausdorff dimension]
    For a set $E\subset\R^n$, we have
    \begin{equation*}
        \dim_H(E)=\sup\{s: H^s(E)=\infty\}=\inf\{s: H^s(E)=0\}
    \end{equation*}
\end{definition}
Before anything, we first check that the $s$-dimensional Hausdorff measure defined above is indeed a measure.
\begin{proposition}
    For $s\geq 0$, the $s$-dimensional measure is indeed a measure.
\end{proposition}
\begin{proof}
    We have that $\mu(\emptyset)=0$, and $\mu(E)\geq 0$ for all $E$. Finally we check the measure is countably additive. For $\{E_j\}_{j\in\mathbb{N}}$ disjoint sets, we consider $E=\cup_jE_j$, as $\delta\to 0$, (or for $\delta$ sufficiently small, given $E_j$'s are disjoint), all the up-to-$\delta$-covers are disjoint, hence
    \begin{equation*}
        H_\delta^s(\cup_jE_j)=\sum_j H_\delta^s(E_j)
    \end{equation*}
    And letting $\delta\to 0$, we get
    \begin{equation*}
        H^s(\cup_jE_j)=\sum_jH^s(E_j)
    \end{equation*}
\end{proof}
\qed

\begin{proposition}
    The following are basic facts about the Hausdorff measure:
    \begin{enumerate}
        \item for $n\in\mathbb{N}$, let $m$ be the $n$-dim Lebesgue measure, there exists a constant $C$ such that
        \begin{equation*}
            C^{-1}H^n(E)\leq m(E)\leq CH^n(E)
        \end{equation*}
        \item $H^s(E)$ is a nonincreasing function of $s$.
        \item For $0\leq s_1<s_2<\infty$
        \begin{equation*}
            \text{either } H^{s_1}(E)=\infty \text{ or } H^{s_2}(E)=0
        \end{equation*}
        \item For $s>n$, and $E\subset\R^n$, we have that
        \begin{equation*}
            H^s(E)=0
        \end{equation*}
        \item For $E\subset\R^n$, and $s\geq 0$, we have that
        \begin{equation*}
            H^s(E)=0 \iff H_\infty^s(E)=0
        \end{equation*}
    \end{enumerate}
\end{proposition}
\begin{example}
    For a set $E\subset\R^n$, we have that the $n$-dimensional Hausdorff measure should agree with the standard Lebesgue measure on $\R^n$. For if $E$ is unbounded, then $m(E)=\infty$, and 
\end{example}


\begin{exercise}
    We have that for $f: A\to R^m, A\subset R^n$, for a fixed $s\geq 0$, and $f$ is Lipschitz with Lipschitz constant $L$, we have that 
    \begin{equation*}
        H^s(f(A))\lesssim_LH^s(A)
    \end{equation*}
\end{exercise}
\begin{proof}
    For any up-to-$\delta$ cover $\{E_j\}$ of $A$, we have $\{f(E_j)\}_j$ is an (up-to-some constant)-$\delta$ cover of $f(A)$, hence 
\end{proof}

\begin{proposition}
    The Hausdorff measure is monotone: for $E_1\subset E_2$, we have that
    \begin{equation*}
        H^s(E_1)\leq H^s(E_2)
    \end{equation*}
\end{proposition}
\begin{proof}
    For $E_1\subset E_2$, for each $\delta$,  an up-to-$\delta$-cover of $E_2$ is also an up-to-$\delta$ cover of $E_1$, and hence taking the infimimum, we get that $H^s(E_1)\leq H^s(E_2)$.
\end{proof}
\qed


\begin{proposition}
    The Hausdorff dimension satisfies that the dimension is a local property:
    \begin{equation*}
        \dim(\cup_jE_j)=\sup_j\dim(E_j)
    \end{equation*}
\end{proposition}
\begin{proof}
    We would like to show that $H^s(\cup_jE_j)=\infty$ if and only if $\sup_jH^s(E_j)=\infty$, and similarly, $H^s(\cup_jE_j)=0$ if and only if $\sup_jH^s(E_j)=0$.

    This is a total of 4 directions. By monotonicity, two directions are shown:
    \begin{equation*}
        \sup_jH^s(E_j)=\infty \Rightarrow H^s(\cup_jE_j)=\infty
    \end{equation*}

    Moreover,
    \begin{equation*}
        H^s(\cup_jE_j)=0\Rightarrow \sup_jH^s(E_j)=0
    \end{equation*}

    Moreover, by $H^s$ being a measure, if we have $\sup_jH^s(E_j)=0$, then all $H^s(E_j)=0$ for all $j$, thus
    \begin{equation*}
        H^s(\cup_jE_j)\leq\sum_jH^s(E_j)=0
    \end{equation*}
    Now it remains to show that \textcolor{red}{what}

\end{proof}


Now we justify the usage of $H^s$, instead of just working $H_\delta^s$.
\begin{exercise}
    For $0\leq s\leq 1, n\geq 2$, we have
    \begin{equation*}
        H_2^s(B_1)=H_2^s(\overline{B_1})=H_2^s(\partial(B_1))
    \end{equation*}
    We see that
    \begin{equation*}
        H_2^s(B)=H_2^s(\overline{B})=2
    \end{equation*}
    Then $H_2^s(\partial B)=0$ if $\overline{B}$ was indeed measurable. But for $0\leq s\leq $, it is more reasonable to cover $\overline{\partial B}$ with bigger covers.
\end{exercise}

Hence we work with $H^s$ to get a Borel regular measure. Recall the following definitions.
\begin{definition}
    A measure $\mu$ is a Borel measure if all Borel sets are $\mu$-measurable. Moreover, $\mu$ is called Borel regular if for any Borel set $A$, there exists another Borel set $B$ such that $B\subset A$, and $\mu(A)=\mu(B)$.
\end{definition}
With our construction, we claim that the Hausdorff measure $H^s$ for any $s>0$ is a Borel regular measure.
\begin{proposition}
    $H_\delta^s$ is a Borel regular measure.
\end{proposition}
\begin{proof}
    We first accept the fact that every Borel set is $H^s$-measurable. We show that $H^s$ is Borel-regular. For a Borel set $A$, we would like to approximate it by ``fattening up'' the covers. For each $n$, let $B_n:=\cup_jE_{n,j}$ be a cover of $A$, and such that $\sum_{j}(diam(E_{n,j}))^s\leq H_\frac{1}{n}^s(A)+\frac{1}{n}$. Then if we take $B=\cap_nB_n$, we have that $A\subset B$, and $H^s(A)=\cap_n H_\frac{1}{n}^s(A)\geq\sum_j(diam(E_{n,j}))^s-\frac{1}{n}\geq \cap_nH_\frac{1}{n}^s(B_n)-\frac{1}{n}$, which by our construction, is $H^s(B)$. Then by monotonicity of $H^s$, we have that
    \begin{equation*}
        H^s(A)=H^s(B)
    \end{equation*}
\end{proof}
\begin{note}
    The countably additivity of $H^s$ comes from the fact that all Borel sets are $H^s$-measurable, and any meausure if countably additive on its measurable sets.
\end{note}

\textcolor{red}{Section 3}
This is part to be typed up. We did Mass distribution principle, which states that if $E$ has that a $r_0$ Frostman measure $\mu$, then $H_{r_0}(E)\geq\mu(E)/C$, and if further we have that $\mu(E)>0$, then $\dim_HE\geq s$.

\begin{enumerate}
    \item Frostman implies positive Hausdorff dimension
    \item definition of support of a measure 
    \item push-forward measure 
\end{enumerate}


This is page 12 on weak convergence of measures.
\begin{definition}[Weak convergence of measures]
    Let $\{\mu_j\}$ be a sequence of locally finite measures (they automatically assign finite measures to all compact sets), and we say $\{\mu_j\}$ converges to $\mu$ weakly if for all $\varphi\in C_c(X)$, we have
    \begin{equation*}
        \lim_{j\to\infty}\int\varphi d\mu_j=\int \varphi d\mu
    \end{equation*}
\end{definition}

Our goal for tonight is to understand the proof of the Frostman Lemma.
\begin{lemma}[Frostman Lemma]
    Assume $E\subset\R^n$ is a compact set with $H^s(E)>0$, then there exists a compactly supported Borel measure $\mu$ with $supp(\mu)\subset E$ and $\mu(E)\gtrsim H_\infty^s(E)$, and such that for all $x\in\R^n, r>0$, we have
    \begin{equation*}
        \mu(B(x,r))\leq r^s
    \end{equation*}
\end{lemma}
\begin{proof}
    
\end{proof}

There are some things we need to establish before the we prove the Frostman lemma. For a set $E\in\R^n$, we use $\mathcal{M}(E)$ to denote the set of finite Borel measures whose support is contained in $E$, i.e. if $\mu\in\mathcal{M}(E)$, we have
\begin{equation*}
    supp(\mu)\subset E, \text{ and }0<\mu(E)<\infty
\end{equation*}

Next we state the ``Bolzano-Weierstrass'' theorem for measures.
\begin{lemma}
    Let $\{\mu_j\}$ be a sequence of locally finite Borel measures on $\R^n$, i.e., for all $K$ compact subset in $\R^n$, we have
    \begin{equation*}
        \sup_{j\in\mathbb{N}}\mu_j(K)<\infty
    \end{equation*}
    Then there exists a subsequence $\mu_{j_k}$ such that as $k\to\infty$, the subsequence converges to $\mu$.
\end{lemma}


\section{Hausdorff dimension of product sets}
\begin{theorem}[Hausdorff dimension of product sets]
    Let $A,B$ be Borel sets, and $s,t\geq 0$, then we have
    \begin{equation*}
        H_\infty^{s+t}(A\times B)\gtrsim_{d_1, d_2}H_\infty^s(A)H_\infty^t(B)
    \end{equation*}
\end{theorem}
\begin{proof}
    We use the theorem that positive Hausdorff meausre if and only if there exists a Frostman measure.
    
    Assume $H_\infty^s(A)>0, H_\infty^t(B)>0$, then there exists $\mu_1, \mu_2$ such that 
    \begin{equation*}
        \mu_1(A)\gtrsim H_\infty^s(A), \mu_2(B)\gtrsim H_\infty^t(B)
    \end{equation*}
    Then we consider any ball $B((x_1, x_2), r)$, we have that
    \begin{equation*}
        B((x_1, x_2), r)\subset B(x_1, r)\times B(x_2, r)
    \end{equation*}
    Hence we have
    \begin{equation*}
        \mu_1\times\mu_2(A\times B)\gtrsim r^{s+t}
    \end{equation*}
    Hence $\mu_1\times\mu_2$ is a Frostman measure on $A\times B$, hence 
    \begin{equation*}
        H_\infty^{s+t}(E)\geq r^{s+t}\gtrsim H_\infty^s(A)H_\infty^t(B)
    \end{equation*}

\end{proof}
\begin{corollary}
    For $A,B$ Borel sets, we have
    \begin{equation*}
        \dim_H(A\times B)\geq \dim_H(A)+\dim_H(B)
    \end{equation*}
\end{corollary}
\begin{proof}
    Once we have $ H_\infty^{s+t}(A\times B)\gtrsim_{d_1, d_2}H_\infty^s(A)H_\infty^t(B)$, it is easy to see, if $\dim_H(A)> s$, and $\dim_H(B)>t$, then we have that $H_\infty^s(A)\geq 0$, and $H_\infty^t(B)\geq 0$, thus $H_\infty^{s+t}(A\times B)\geq 0$, hence we have
    \begin{equation*}
        \dim_H(A\times B)\geq s+t
    \end{equation*}
\end{proof}
\qed

\begin{corollary}
    We also have that
    \begin{equation*}
        \dim_H(A\times B)\leq\dim_HA+\overline{\dim_M}B
    \end{equation*}
\end{corollary}
\begin{proof}
    If we assume that $\dim_H(A)<s, \dim_H(B)\leq\overline{\dim_M}(B)<t$, then we have
    \begin{equation*}
        \dim_H(B)<t, i.e. H_\infty^t(B)=0
    \end{equation*}
    \begin{equation*}
        \dim_\infty(A\times B)
    \end{equation*}
\end{proof}



\section{Riesz Energies}
We revisit the question regarding projections.
Assume $E$ is compact,do we always have
\begin{equation*}
    \dim_H(\pi_e(E))=\min\{\dim_H(E), 1\}
\end{equation*}
for some $e\in S^1$ or for all $e\in S^1$.

The important implication is we would like to transfer an $s$-dimensional Frostman measure on $E$ to an $s$-dimensional Frostman measure(the push-forward measure) on $\pi_e(E)$, i.e. we would like to have if $\mu(B(x,r))\lesssim r^s$ for $supp(\mu)\subset E$,
\begin{equation*}
    \mu_{\pi_e}(B(x,r))\lesssim r^s
\end{equation*}
where $supp(\mu_{\pi_e})\subset \pi_e(E)$.

\begin{definition}[Riesz potential and energy of measures]
    Let $0\leq s\leq d$, and let $\mu$ be a Borel measure on $\R^d$. The $s$-dimensional Riesz potential of the measure $\mu$ at a particular point is
    \begin{equation*}
        V_s(\mu)(x)=\mu\ast k_s(x)=\int\frac{1}{|x-y|^s}d\mu(y)
    \end{equation*}
    where $k_s$ is the $s$-dimensional Riesz kernel 
    \begin{equation*}
        k_s(x)=\frac{1}{|x|^s}
    \end{equation*}
    And the $s$-dimensional Riesz energy of the measure $\mu$ is given by integrating the potential:
    \begin{equation*}
        I_s(\mu)=\int V_s(\mu)d\mu(x)=\int\int\frac{1}{|x-y|^s}d\mu(x)d\mu(y)
    \end{equation*}
\end{definition}

The reason for introducing the Riesz energies and potentials is because they almost carry the same information as a measure being $s$-Frostman.
\begin{proposition}[Finite energy if and only if $s$-Frostman]
    Let $\mu\in M(\R^d)$ be $s$-Frostman, then for all $0\leq t<s$, we have $\|V_t(\mu)\|_\infty<\infty$, and
    \begin{equation*}
        I_t(\mu)<\infty
    \end{equation*}
    Conversely, if we have finite $s$-Riesz energy, $I_s(\mu)<\infty$, then there exists a subset $B\subset\R^d$, that we have
    \begin{equation*}
        \mu(B)>0, \mu\vert_B \text{ is $s$-Frostman}
    \end{equation*}
\end{proposition}
\begin{proof}
    \textcolor{red}{computation, should do this}
\end{proof}

As an immediate corollary, we establish a relationship between finite energy of a measure with positive Hausdorff dimension.
\begin{corollary}[Finite energy and positive Hausdorff measure]
    Let $E\subset\R^d$, then $\dim_H(E)>s\geq0$, then there exists a measure $\mu$ such that $\mu\in M(E)$, and 
    \begin{equation*}
        I_s(\mu)<\infty
    \end{equation*}
    Conversely, if there exists a $\mu\in M(E)$ such that $I_s(E)<\infty$, then we have
    \begin{equation*}
        \dim_H(E)\geq s
    \end{equation*}
\end{corollary}

\begin{note}
    Justifying that the measure has finite energy for $H^1$ almost every $e$ is the strategy for proving Marstrand's projection theorem.
\end{note}
Now we prove the Marstrand's projection theorem.
\begin{theorem}[Marstrand's projection theorem]
    Let $E\subset\R^2$ be Borel, then
    \begin{equation*}
        \dim_H\pi_e(E)=\min\{\dim_H(E), 1\}
    \end{equation*}
    for $H^1$ almost every $e\in S^1$.
\end{theorem}


\section{Rectifiable sets}
There is actually a second part of the statement of Marstrand's projection theorem. 
\begin{theorem}
    If $\dim_H(E)>1$, then for $H^1$ almost every $e$, we have
    \begin{equation*}
        H^1(\pi_e(E))>0
    \end{equation*}
\end{theorem}
To prove this, we will use tools in section 6. However, we can ask the following question now:
\begin{problem}
    If $E\subset\R^2$ is compact, can we always have
    \begin{equation*}
        H^1(\pi_e(E))>0?
    \end{equation*}
\end{problem}
\begin{note}
We will answer this with a positive answer if $E$ is 1-rectifiable, and $0<H^1(E)<\infty$. And we will answer this with a negative answer, as our last theorem of the section, the Besicovitch theorem.
\end{note}
\begin{theorem}
    Let $E\subset\R^2$ be a purely 1-unrectifiable set with $0<H^1(E)<\infty$, then for $H^1$ almost every $e\in S^1$, we have
    \begin{equation*}
        H^1(\pi_e(E))=0
    \end{equation*}
\end{theorem}


We define the densities at a particular point.
\begin{definition}[$s$-dimensional density]
    Let $A\subset\R^d$, and $H^s(A)<\infty$, for $s\geq 0$.
    The upper and lower $s$-dimensional densities of $A$ at $x$ is defined to be
    \begin{equation*}
        \Theta^{s,*}(A,x)=\limsup_{r\to 0}\frac{H^s(A\cap B(x,r))}{r^s}, \Theta_*^s(A,x)=\liminf_{r\to 0}\frac{H^s(A\cap B(x,r))}{r^s}
    \end{equation*}
\end{definition}

\begin{note}
    The following should be interpreted just as the Lebesgue point density theorem.
\end{note}
The following two propositions are just like Lebesgue density theorems, and I shall call them the analogs of Hausdorff  density theorems. This implies that depending on whether $x$ is in the set or not, the density at that point is either positive or 0.
\begin{proposition}[Hausdorff density theorem]
    For $H^s$ almost all $x\in A$.
    \begin{equation*}
        1\leq\Theta^{s,*}(A,x)\leq 2^s
    \end{equation*}
\end{proposition}
And it is what we expected for $x\not\in A$.
\begin{proposition}
    For $H^s$ almost all $x\in\R^d\setminus A$, we have
    \begin{equation*}
        \Theta^{s,*}(A,x)=0
    \end{equation*}
\end{proposition}

\subsection{Rectifiable sets}
Let's define the two types of sets that answer oppositely to our positive projection theorem.
\begin{definition}[rectifiable]
    Let $0<n<d$ be integers, and let $E\subset\R^d$, and we say that $E$ is $n$-rectifiable if for $H^n$ almost all of $E$ can be covered by Lipschitz images of $\R^n4$. In other words, there exists a set of countable Lipschitz maps $\{f_j\}, f_j:\R^n\to\R^d$ such that
    \begin{equation*}
        H^n\left( E\setminus \bigcup_jf_j(\R^n)\right)=0
    \end{equation*}
\end{definition}
\begin{definition}[n-unrectifiable]
    $E$ is called purely $n$-unrectifiable if for all Lipschitz map $f:\R^n\to\R^d$, we have
    \begin{equation*}
        H^n(E\cap f(\R^n))=0
    \end{equation*}
    (This means that $E$ does not intersect nontrivially with any $f(\R^n)$). 

    Alternatively, $E$ is $n$-unrectifiable if and only if for all $n$-rectifiable sets $R\subset\R^d$, we have
    \begin{equation*}
        H(E\cap R)=0
    \end{equation*}
\end{definition}
The following asserts that any $E\subset\R^d$ with $H^n(E)<\infty$ can be decomposed into $n$-rectifiable sets, and purely $n$-unrectifiable sets.
\begin{theorem}
    Let $0<n<d$, and let $E\subset\R^d$, and $H^n(E)<\infty$, and then there exists $R\subset\R^d$, $n$-rectifiable, such that $E$ can be written into disjoint unions of rectifiable and unrectifiable portions:
    \begin{equation*}
        E=[E\cap R]\cup U
    \end{equation*}
    where $U=E\setminus R$ is purely $n$-unrectifiable, and $E\cap R$ is $n$-rectifiable.
\end{theorem}



\subsection{Projections of rectifiable sets}





\subsection{Projection of unrectifiable sets}
We now talk about projections of unrectifable sets. 
\begin{theorem}[Besicovitch]
    Assume $E\subset \R^2$ is Borel purely 1-unrectifiable 1-set. Then we have 
    \begin{equation*}
        H^1(\pi_e(E))=0
    \end{equation*}
    for $H^1$ almost every $e\in S^1$.
\end{theorem}
We first define the notion of high multiplicity and high density for directions $e\in S^1$.

\begin{definition}[High Multiplicity]
    We say a direction $e$ is a direction of high multiplicity at $x$, denoted $e\in H_x$, if
    \begin{equation*}
        |E\cap l_e(x)\cap B(x,r)|\geq 2, \text{ for all } r>0
    \end{equation*}
    Alternatively, for a fix $r>0$, we can denote the above as $H_x(r)$, then we have
    \begin{equation*}
        H_x=\bigcap_{r>0}H_x(r)
    \end{equation*}
\end{definition}
The condition ``for all $r>0$'' implies that there are points in $E$ that are in $l_e(x)$ that are infinitely close to $x$.

\begin{definition}[High density]
    We say $e$ is a direction of high density at $x$, denoted that $e\in D_x$, if for all $r_0, M, \epsilon>0$, there exists a radius $0<r<r_0$, and an arc $J\subset S^1$, and $e\in J$, $0<\sigma(J)<\epsilon$, we have
    \begin{equation*}
        \frac{\mu(C(x,J)\cap B(x,r))}{r}\geq M\sigma(J)
    \end{equation*}
    where $\mu=\frac{1}{|H^s(E)|}H^1\vert_E$, and $C(x, J)$ is the cone $C(x,J)=\bigcup_{e\in S}l_e(x)$.

    Alternatively, for fixed $r_0, M, \epsilon>0$, we have
    \begin{equation*}
        D_x=\bigcup_{r_0, \epsilon, M>0}D_x(r_0,\epsilon, M)
    \end{equation*}
\end{definition}
\begin{note}
    Both high multiplicities and high denstiies at $x$ mean that there are plenty of points in $E$ that are arbitrarily close to $x$, contained in an arbitrarily narrow cones around $l_e(x)$. This could give intuition that $E$ has very small projections.
\end{note}

\begin{note}
    \textbf{Proof Sketch} We will show that for $H^1_E$ almost every $x\in E$, and for almost all  $H^1_S$ directions $e$, we have that $e$ is either high multiplicity or high density. Then we will show that points with almost all directons that are either of high density or high multiplicity will have small projections in almost all directions.
\end{note}

\begin{lemma}[High density or high multiplicity]
    For $H^1$ almost every $x\in E$, and $H^1$ almost every $e\in S^1$ is either a direction of high multiplicity ($e\in H_x$) or high density at $x$ ($e\in D_x$).
\end{lemma}
\begin{proof}
    We will specify what those points are and show that $H^1(\text{not these points})$ are of measure 0. In other words, for any fixed $r_0, \epsilon, M>0$, we show that for $\mu$ almost every $x\in E$, we have
    \begin{equation*}
        \sigma(S^1\setminus[H_x(r_0)\cup D_x(r_0, \epsilon, M)])=0
    \end{equation*}
    Since $\mu$ almost every $x\in E$, satisfies 
    \begin{equation*}
        \frac{\mu(B(x,r)\cap C(x,J))}{r}\geq c\sigma(J)
    \end{equation*}
    for arbitrary arc $J\subset S^1$, it suffices to show that for almost every $e\in S^1$, we have either
    \begin{equation*}
        \Theta^{1,*}(H_x(r_0), e)>0 \text{ or  } \Theta_x^1(D_x(r_0, \epsilon, M))>0
    \end{equation*}
    And note that for $\sigma$ almost  everywhere $e\in S^1\setminus [H_x(r_0)\cup D_x(r_0, \epsilon, M)]$, we have
    \begin{equation*}
        \Theta^{1}(H_x(r_0), e)=\Theta^1(D_x(r_0, \epsilon, M), e)=0
    \end{equation*}
    Then we have 
\end{proof}

t