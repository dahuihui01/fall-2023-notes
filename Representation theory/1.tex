\chapter*{What is representation theory}

\section{Introduction}


We would like to represent abstract algebraic concepts using linear objects. We will start with an example of a specific type of representation.
\begin{definition}[Representation on a vector space]
    A representation $\pi$ of a group $G$ on a $\K$-vector space $V$ is a mapping:
    \begin{equation*}
        \pi: G\to Aut(V)=GL(V)
    \end{equation*}
    It assigns every element in $G$ with an invertible linear map on $V$.
\end{definition}

We will categorize some interesting features of a representation.
\begin{definition}[faithful representation]
    A representation is called faithful if it is injective,i.e., for $g_1, g_2\in G$, we have
    \begin{equation*}
        \pi(g_1)=\pi(g_2) \text{ implies } g_1=g_2
    \end{equation*}
\end{definition}
\begin{remark}
    This means that there is ``no loss.''
\end{remark}
\begin{definition}[Trivial representation]
    The trivial representation of $G$ on $V$ is such that 
    \begin{equation*}
        \pi(g)=I
    \end{equation*}
    Or $\pi(g)v=v$ for all $v\in V$.
\end{definition}
\begin{example}
    For the $\R$-vector space $\C^4$, we have a faithful representation of a group order 8, with each $g$ sent to each one of the following linear maps:
    \begin{equation*}
        \begin{bmatrix}
            1& 0\\
            0& 1\\
        \end{bmatrix}
        \begin{bmatrix}
            1& 0\\
            0& -1\\
        \end{bmatrix}
        \begin{bmatrix}
            0& 1\\
            1& 0\\
        \end{bmatrix}
        \begin{bmatrix}
            0& -1\\
            1& 0\\
        \end{bmatrix}
        \begin{bmatrix}
            i& 0\\
            0& i\\
        \end{bmatrix}
        \begin{bmatrix}
            i& 0\\
            0& -i\\
        \end{bmatrix}
        \begin{bmatrix}
            0& i\\
            i& 0\\
        \end{bmatrix}
        \begin{bmatrix}
            0& -i\\
            i& 0\\
        \end{bmatrix}
    \end{equation*}
\end{example}

Now we introduce ``equivalent'' representations.
\begin{definition}[Intertwiner]
    For two representations $(\pi_1, V_1), (\pi_2, V_2)$ of $G$, a linear isomorphism $\phi: V_1\to V_2$ is called an Intertwiner if it ``intertwines'' the action of $G$:
    \begin{equation*}
            \phi(\pi_1(g)v)=\pi_2(g)(\phi(v))
    \end{equation*}
\end{definition}
\begin{remark}
    We can think of $\phi$ as the isomorphism between $V_1$ and $V_2$ such that the following diagram commutes:
    \begin{equation*}
        \begin{tikzcd}
            & V_1 \arrow{d}{\phi} \arrow{r}{\pi_1(g)} & V_1 \arrow{d}{\phi}\\
            & V_2 \arrow{r}{\phi_2(g)} & V_2
        \end{tikzcd}
    \end{equation*}
\end{remark}

Now we find the building blocks of representations: invariant subspaces, and irreducible representations.
\begin{definition}[Invariant subspaces]
    Let $(\pi, V)$ be a representation of $G$, and let $K\subset V$, then $K$ is called a $G$-invariant subspace if for all $w\in W$,
    \begin{equation*}
        \pi(g)w\in W, \text{ for all }g\in G
    \end{equation*}
\end{definition}
\begin{definition}[Irreducible representations]
    A representation $(\pi, V)$ of $G$ is called irreducible if it contains no proper $G$-invariant subspaces. (And is call totally reducible if it can be written as a direct sum of irreducible subspaces).
\end{definition}

