\chapter{Lecture 1}
Here we go.

\subsection{Course Overview and Logistics}
Some administrative things.
OH are Monday, Fridays 1:45 to 2:45, Wednesdays 12:45-1:45 in Evans 811.

\textbf{Textbook}: an introduction to functional analysis by Conway.
We will be talking about operators on Hilbert spaces, and more generally, Banach spaces, and Frechet spaces (defined by a countable numer of seminomrs).

\begin{remark}
    Let $\mathcal{H}$ be a Hilbert space, then the dual space $\mathcal{H}^*$ is itself. $\mathcal{H}=\mathcal{H}^*$.    
    Hilbert spaces are the best spaces to work with. They are self-dual, and identified with themslves.
\end{remark}
Then in the next section, we will look at groups, motivated by their actions on Banach spaces, connected with Fourier transforms. 


\subsection{Motivation}
Let $X$ be a compact Hausdorff space. Let $C(X)=\{f:X\to\R, f \text{ continuous}\} $ be the algebra of continuous functions on $X$ mapping in to $\R$ or $\mathbb{C}$.
Define the norm as the sup norm $\|\cdot\|_{L^\infty}$.

We will develop the spectral theorem of operators on the Hilbert space, i..e self-adjoint operators can be diagonalized.

If $T$ is a self-adjoint operator on a Hilbert space, then we take the product of $T$ (polynomials of $T$), let $C^*(T, I_\mathcal{H})$ be the sub-algebra of operators generated by $T$ and $I$ the identity operator, then take the closure, i.e. making it closed in the operator norm.

\begin{remark}
    The $*$ is to remind us, $T$ is self-adjoint and when you take the adjoint and generate with it, it gets back into the same space.
\end{remark}

\begin{proposition}
    We have the next two algebra isomorphic to each other.
    \begin{equation}
        C^*(T, I_\mathcal{H})\cong C(X)
    \end{equation}
\end{proposition}
This is what we are aiminig for. We can generalize this even further to finitely many self-adjoint operators, in some sense, we are diagonalizing finitely many operators at the same time. If $T_1,..., T_n$ is a collection of self-adjoint operators on $\mathcal{H}$, and such all commute with each other, then we also have
\begin{equation}
    C^*(T_1,..., T_n, I_\mathcal{H})\cong C(X)
\end{equation}

\subsection{Groups}

Let $G$ be a group, $B$ be a Banach space, for example, groups of automorphisms. Let
\begin{equation*} 
    Aut(B)=\{T:  T \text{is isometric, onto, invertible on } B\}
\end{equation*}

\begin{definition}
    Suppose that $\alpha$ is a group homomorphisms, and $\alpha: G\to Aut(B)$, is called a representation on $B$ or an action of the group $G$ on $B$.
\end{definition}


Then we can consider the subalgebra $\mathcal{L}(B)$, consisting of the bounded linear operators on $B$, generated by
\begin{equation*}
    \{\alpha_x:x\in G\}
\end{equation*}
\begin{remark}
    The identity on $G$ should be mapped into the identity operator on $B$, hence no need to include it.
\end{remark}

Elements of the form $\Sigma_{z_x}\alpha_x, z_x\in\mathbb{C}$, (where $\Sigma$ is a finite sum.)

Let's introduce,
$f\in C_c(G)$ are functions with compact support and in discrete groups, imply they are of finite support.

\begin{equation*}
    \sum_{x\in G}f(x)\alpha_x=\alpha_f
\end{equation*}
note for except finitely many $x$, $f(x)=0$.

Let $f,g\in C_c(G)$, then for 
\begin{equation*}
    \alpha_f\alpha_g=(\sum f(x)\alpha_x)(\sum g(y)\alpha_y)=\sum_{x,y}f(x)g(y)\alpha_x\alpha_y=\sum_{x,y}f(x)g(y)\alpha_{xy}
\end{equation*}
The last inequality follows from $\alpha$ being a group homomorphism. And the sums are finite hence are able to exchange the orders.
We further have,
\begin{equation*}
    \alpha_f\alpha_g=\sum_x\sum_yf(x)g(x^{-1}y)\alpha_y=\sum (f\ast g)(y)\alpha_y
\end{equation*}
where we define $f\ast g(y)=\sum f(x)g(x^{-1}y)$ as the convolution operator.


We get
\begin{equation*}
    \alpha_f\alpha_g=\alpha_{f\ast g}
\end{equation*}
This is how we define convolution on $C_c(G)$
Notice we have, by $\|\alpha_x\|=1$,
\begin{equation*}
    \|\alpha_f\|=\|\sum f(x)\alpha_x\|\leq\sum|f(x)|\|\alpha_x\|=\sum|f(x)|=l^1(f)=\|f\|_{l^1}
\end{equation*}
It is therefore, easy to check

\begin{equation*}
    \|f\ast g\|_{l^1}\leq\|f\|_{l^1}\|g\|_{l^1}
\end{equation*}
We get $l^1(G)$ is an algebra with ??

For $G$ commutative, it is easily connected with the Fourier transform.

Consider $l^2(G)$ with the counting measure on the group. For $x\in G$, let $\xi\in l^2(G)$ define $\alpha_x\xi(y)=\xi(x^{-1}y), \alpha_x$ being unitary.
$l^1(G)$ acts on operators in $l^2(G)$ via $\alpha$.


If $G$ is commutative, then we have
\begin{equation*}
    \overline{\alpha_{l^1(G)}}\cong C(X)
\end{equation*}
where $X$ is some compact space.
Note that $C_c(G)$ operators on $l^2(G)$, and $\|\alpha_f\|\leq\|f\|_{l^1}$.

\newpage
\section{Lecture 2}
Let's do some math.

Let $X$ be a Hausdorff compact space, and let $C(X)$ denote the space of continuous functions defined on $X$. This is an algebra. You can multiply them, associatively and commutatively. We equip it with a norm $\|\cdot\|_{L^\infty}$. Note $X$, by assumption, is a normal space, you could have continuous functions mapped to 1 on one subset, 0 to the other subset. Hence there are many elements from $C(X)$.

\begin{definition}[Normed Algebra]
    Let $\mathcal{A}$ be an algebra on $\R$ or $\mathbb{C}$, is a normed algebra if it has a norm $\| \|$, as a vector space, such that for for $a,b\in\mathcal{A}$, we have
    \begin{equation*}
        \|ab\|\leq\|a\|\|b\|
    \end{equation*}
    The above is called submultiplicity.
\end{definition}
\begin{definition}[Banach Algebra]
    A Banach Algebra is a normed algebra that is complete in the metric space from the norm.
\end{definition}


Given $x\in X$, define $\varphi_x: C(X)\to\mathbb{C}$ the evaluation map such that
\begin{equation*}
    \varphi_x(f)=f(x)
\end{equation*}

$\varphi_x$ is an algebra homomorphisms between $C(X)\to\R$ or $C(X)\to\mathbb{C}$. This simply implies
\begin{equation*}
    \varphi_x(f+g)=(f+g)(x)=f(x)+g(x), \varphi_x(fg)=(fg)(x)=f(x)g(x)
\end{equation*}

We now make the note that, $C(X)$ has an identity element, which is the constant function $1$, under multiplication. Hence $C(X)$ is a unital algebra. Note that $\varphi_x$ defined above is a unital homomorphism, meaning that it sends identity to identity.

Note $\varphi_x$ is also a multiplicative linear functional, also unital.

\begin{proposition}
    Every multiplicative linear functional on $C(X)$ is of the form $\varphi_x$ for some $x\in X$.
\end{proposition}
\begin{proof}
    Main Claim: given a multiplicative linear functional $\varphi$, there exists a point $x_0$ and if we have some $f\in C(X)$, we have $\varphi(f)=0$, then we have $f(x_0)=0$. To prove this claim, we need compactness. Suppose the contrary of the claim. Suppose that for each $x\in X$, there is an $f_x\in C(X)$ such that $f(x)\neq 0$, but $\varphi(f)=0$.

    Set $g_x=\overline{f}_xf_x$, then we have $g_x(x)>0$, but $\varphi(g_x)=\varphi(f_x)\varphi(\overline{f}_x)=0$, then there is an open set $O_x$ such that $x\in O_x$, and $g_x(y)>0$ for all $y\in O_x$. Now by compactness, there is $x_1, ..., x_n$ such that $X=\bigcup_{j=1}^nO_{x_j}$, let $g=g_{x_1}+... g_{x_n}$, then we have $g(y)>0$ for all $y\in X$, and $\varphi(g)=0$. Note that $g$ is a continuous function, and $g$ is invertible, and also $re(\frac{1}{g})\in C(X)$, but we also have
    \begin{equation*}
        \varphi \left(g\cdot\frac{1}{g}\right)=1
    \end{equation*}
    Hence we've reached a contradiction.
    Then there exists $x_0\in X$ such that if $\varphi(f)=0$, this means $f(x_0)=0$. For any $f$, consider $f-\varphi(f)\cdot 1$, apply $\varphi$, we have
    \begin{equation*}
        \varphi(f-\varphi(f)\cdot 1)=0, \text{ this implies there exists } x_0, \text{ such that } (f-\varphi(f)1)(x_0)=0
    \end{equation*}
    This implies $f(x_0)=\varphi(f)$ which implies $\varphi(f)=\varphi_{x_0}(f)$.
\end{proof}

For any unital commutative algebra $\mathcal{A}$ and let $\widehat{\mathcal{A}}$ be the set of unital homomorphisms of $\mathcal{A}$ into the field.

For $\mathcal{A}=C(X)$, and $\varphi\in\widehat{\mathcal{A}}$. 
\begin{definition}
    For any unital commutative algebra $\mathcal{A}$ and let $\widehat{\mathcal{A}}$ be the set of unital homomorphisms of $\mathcal{A}$ into the field.
\end{definition}
\begin{remark}
    We have $|\varphi(f)|\leq\|\varphi\|\|f\|_{L^\infty}$, since $\varphi$ is unital, we have $\|\varphi\|=1$.
\end{remark}

Thss is not always true for normed algebra,
Let
\begin{equation*}
    \mathcal{A}:= Poly\subset C([0,1])
\end{equation*}
We define $\varphi(p)=p(2)$, $p$ is a polynomial. This is not continuous, nor is the $\|\varphi\|=1$.


\begin{proposition}
    If $\mathcal{A}$ is a unital commutative Banach algebra, and if $\phi\in\widehat{\mathcal{A}}$, then we have $\|\varphi\|=1$.
\end{proposition}

The word ``unital'' is key here.

\begin{proposition}
    Let $\mathcal{A}$ be a unital Banach algebra (not necessarily commutative), then if $a\in\mathcal{A}$, and $\|a\|< 1$, then we have
    \begin{equation*}
        1_\mathcal{A}-a \text{ is invertible in } \mathcal{A}
    \end{equation*}
\end{proposition}
\begin{proof}
    For this, we use completeness.
    $\frac{1}{1-a}=?\sum_{n=0}^\infty a^n, a^0=1_\mathcal{A}$
    You could look at the partial sums. $S_m=\sum_{n=0}^ma^n$, you want to show that $\{S_m\}$ is a Cauchy sequence, and use completeness of Banach algebras. $\lim_{m\to\infty}S_m=\frac{1}{1-a}$.
    
    To prove this is a cauchy sequence:
    \begin{equation*}
        \|S_n-S_m\|=\|\sum_{j=m+1}^na^j\|\leq\sum_{m+1}^n\|a^j\|\leq\sum_{m+1}^n\|a\|^j
    \end{equation*}
    And the fact that $\|a\|\leq 1$, we have the sum bounded by $\epsilon$, hence $\{S_n\}$ is a Cauchy sequence. Let $b=\sum_{n=0}^\infty a^n$, we want to show that $b(1-a)=1$.
    \begin{equation*}
        b(1-a)=\lim_{n\to\infty}{S_n}(1-a)=\lim_{n\to\infty}\left(\sum_{n=0}^\infty a^n\right)(1-a)=\lim_{n\to\infty}(1-a^{n+1})=0
    \end{equation*}
    The last inequality follows from $\|a^{n+1}\|\leq\|a\|^{n+1}\to 0$.
\end{proof}


\section{Lecture 3}
We now begin.

Let $\mathcal{A}$ be a unital Banach algebra, and if $a\in\mathcal{A}$ and $\|a\|<1$, then we have $(1-a)$ has an inverse and if $\mathcal{A}=\mathcal{B}(B)$, where $B$ is some Banach space, then $T\in\mathcal{A}$, and $\|T\|<1$, then we have
\begin{equation*}
    (1-T)^{-1}=\sum T^n
\end{equation*} 

The above is called the Newmann series.

Now we have the following corollary.
\begin{corollary}
    If $a\in\mathcal{A}$ and $\|1-a\|<1$, then $a$ is invertible.
\end{corollary}
\begin{proof}
    $a=1-(1-a)$.
\end{proof}


\begin{proposition}
    The set of invertible elements of $\mathcal{A}$ is an open subset of $\mathcal{A}$.
\end{proposition}
\begin{proof}
    The open ball about 1 consists of invertible elements. If $d$ is any invertible element, then we define $a\mapsto da$. This map is continuous, i.e. it is the left representation $L_b(a)=ab$ for all $a\in\mathcal{A}$. If $d$ is invertible, then the inverse is also continuous, hence it is a homeomorphism of $\mathcal{A}$ onto itself.

    Denote the unit ball about $1$ as $B_1(1)$, and let $d$ be some invertible element, under $L_d$, homeomorphism, $O\mapsto d\cdot O$, this set is open, and consists of invertible elements. We take the union of all these elements, which give us an open set including every invertible elements.
\end{proof}
\qed

\begin{proposition}
    Let $C(X)$ be the unital Banach algebra, and for $f\in C(X)$, we have $\alpha\in Range(f)$ if and only if $(f-\alpha\cdot 1)$ is not invertible.
\end{proposition}
\begin{proof}
Let $f\in C(X)$, and if $\alpha\in$ range of $f$, so $\alpha=f(x_0)$ for some $x_0$. then
\begin{equation*}
    (f-\alpha\cdot 1)(x_0)=0
\end{equation*}
Hence $f(-\alpha\cdot 1)$ is not invertible.
Conversely, if we have $f-\alpha 1$ is not invertible, then there exists $x_0\in X$ such that
\begin{equation*}
    (f-\alpha\cdot1)(x_0)=0
\end{equation*}
Hence $f(x_0)=\alpha$, i.e., $\alpha\in$ range of $f$.
\end{proof}
\qed


\begin{definition}[spectrum of an element]
    For any unital algebra $\mathcal{A}$ over some field $\mathbb{F}$, for any $a\in\mathcal{A}$, the set
    \begin{equation*}
        \{\lambda\in\mathbb{F}: a-\lambda 1_\mathcal{A}\text{ is not invertible }\}
    \end{equation*}
    is called the spectrum of $a$, denoted as $\sigma(a)$.
\end{definition}

Interpret this in our familiar linear map: $\lambda$ is called an eigenvalue, i.e. is in the spectrum of $T$ if we have $T-\lambda I$ is not invertible.

\begin{proposition}
    Let $\mathcal{A}$ be a unital Banach algebra, and let $a\in\mathcal{A}$, then if $\lambda\in\sigma(a)$, then
    \begin{equation*}
        |\lambda|\leq\|a\|
    \end{equation*}
\end{proposition}
\begin{proof}
    Suppose $|\lambda|>\|a\|$, then $\lambda\neq 0$, then
    \begin{equation*}
        a-\lambda\cdot 1=-\lambda(1-\frac{a}{\lambda})
    \end{equation*}
    And by assumption, $\|a/\lambda\|\leq 1$, hence $(1-a/\lambda)$ is invertible. Hence $a-\lambda\cdot 1$ is invertible (product of two invertible elements),  meaning $\alpha\not\in\sigma(a)$.
\end{proof}
\qed

\begin{proposition}
    Let $\varphi$ be a multiplicative linear functional on $\mathcal{A}$, i.e. $\varphi\in\widehat{\mathcal{A}}$, and then $\varphi(a)\in\sigma(a)$, and we have
    \begin{equation*}
        |\varphi(a)|\leq\|a\|, \|\varphi\|=1
    \end{equation*}
\end{proposition}
\begin{proof}
    $\varphi(a-\varphi(a)\cdot 1)=0$.
    Hence $a-\varphi(a)1$ is not invertible.
\end{proof}
\qed


\begin{proposition}
    $\sigma(a)$ is a closed subset of $\mathbb{R}, \mathbb{C}$.
\end{proposition}
\begin{proof}
    Define the map $\phi: \lambda\mapsto a-\lambda1$, the map $\phi$ is continuous (multiplication and subtraction are both continuous). We know the set of invertible elements of $\mathcal{A}$ is open, hence
    \begin{equation*}
        \sigma(a)=\phi^{-1}(\text{ noninvertible})=\phi^{-1}(\mathcal{A}\setminus\text{ invertible })
    \end{equation*}
    Or simply,
    \begin{equation*}
        \sigma(a)=(\phi^{-1}(\text{ invertible }))^c
    \end{equation*}
    Hence the spectrum of an element is closed.
\end{proof}
\qed


Let $\varphi\in\widehat{\mathcal{A}}$ then $\|\varphi\|=1$. So $\widehat{\mathcal{A}}$ is a subset of the unit ball of $\mathcal{A}'$, which denotes the dual vector space of continuous linear transformations.


On $\mathcal{A}'$, we can equip the weak-* topology, i.e. the weakest topology, making the map $\psi\mapsto\psi(a)$ continuous. 

\begin{proposition}
    $\widehat{\mathcal{A}}$ is closed for the weak-* topology.
\end{proposition}
\begin{proof}
    let $\{\varphi_\lambda\}$ be a net of elemnts of $\widehat{\mathcal{A}}$, that converges to some $\psi\in\mathcal{A}'$ in the weak-* topology, i.e., for every $a\in\mathcal{A}$, $\varphi_\lambda(a)\to\psi(a)$ for all $a\in\mathcal{A}$.

    Then $\varphi(a,b)=\lim\varphi_\lambda(ab)=\lim\varphi_\lambda(a)\varphi_\lambda(b)=\varphi(a)\varphi(b)$.

    $\varphi(1)=\lim(\varphi_\lambda(1))=\lim 1=1$.
\end{proof}

\begin{theorem}[Alaoglu's theorem]
    For any normed vector space $V$, the closed unit ball of $V'$ is compact in the weak-* topology. 
\end{theorem}
As an immediate corollary, we have the following.
\begin{corollary}
    $\widehat{\mathcal{A}}$ is compact with respect to the weak-* toplogy.
\end{corollary}
\begin{proof}
    $\widehat{\mathcal{A}}$ is a closed subset of a compact set, hence is also compact.
\end{proof}
\qed

Let $\mathcal{A}=C(X)$, and $\widehat{\mathcal{A}}$, we define $x\mapsto\varphi_x$ is a bijection. The weak-* topology in $\widehat{\mathcal{A}}$ makes $\varphi_x\mapsto\varphi_x(f)=f(x)$ continuous. Such $x\mapsto\varphi_x$ is a homomorphism of $X$ onto $\mathcal{A}$.

For $\mathcal{A}$ unital Banach algebra, commutative, for any $a\in\mathcal{A}$, define
\begin{equation*}
    \widehat{a}\in C(\widehat{\mathcal{A}}), \widehat{a}(\varphi)=\varphi(a)
\end{equation*}

\begin{proposition}
    The map $a\mapsto\widehat{a}$ is a unital algebra homomorphism from $\mathcal{A}$ into $C(\mathcal{A})$.
\end{proposition}
\begin{proof}
    we have
\begin{equation*}
    \widehat{ab}(\varphi)=\varphi(ab)=\varphi(a)\varphi(b)=\widehat{a}(\varphi)\widehat{b}(\varphi)=(\widehat{a}\widehat{b})(\varphi)
\end{equation*}

Hence
\begin{equation*}
    (\widehat{ab})=\widehat{a}\widehat{b}, \widehat{(a+b)}=\widehat{a}+\widehat{b}, \widehat{1_a}=1
\end{equation*}
\end{proof}
\qed


\section{Lecture 4}
Today we talk about the structure of $\widehat{l^1(S)}, \widehat{l^1(G)}$, where $S,G$ are semigroups and groups, and how they naturally identify with the unit disk $\D$, and the unit circle $\T$.

Let $S$ be a commutative discrete semigroups, for example $\N\cup\{0\}$, and $f\in C_c(S)$, then we can write $f=\sum_{x\in S}f(x)\delta_x$, where we define $\delta_x\delta_y=\delta_{xy}$. Note that $C_c(S)$ is dense in $l^1(S)$.
\begin{definition}[Convolution]
    Take any $f,g\in C_c(S)$, we consider the following:
    \begin{equation*}
        \sum_{x\in S}f(x)\delta_x\sum_{x\in S}g(y)\delta_y=\sum_{x\cdot y}\delta_{xy}=\sum_{z\in S}\left(\sum_{xy=z}f(x)g(y)\right)\delta_z
    \end{equation*}
    where we define the convolution between two functions
    \begin{equation*}
        f\ast g(z)=\sum_{x,y, xy=z}f(x)g(y)
    \end{equation*}
\end{definition}

And under this convolution operation, we have $l^1(S), \ast$ as a Banach algebra.
\begin{example}
    If we consider polynomials of the form $f(x)=\sum_{n=0}^\infty f(n)x^n$, and consider the operation between two polynomials
    \begin{equation*}
        \left(\sum f(m)x^m\right)\left(\sum g(n)x^n \right)=\sum_{p}\left(\sum_{m+n=p}f(m)g(n)x^p \right)=\sum_{p}(f\ast g)(p)
    \end{equation*}
\end{example}

And let $f\in C_c(S)$, where $S=\N$. we define $\|f\|_{l^1}=\sum_{x\in S}|f(x)|$.

It is easy to check we have
\begin{equation*}
    \|f\ast g\|_{l^1}\leq\|f\|_{l^1}\|g\|_{l^1}
\end{equation*}

We let $\mathcal{A}=l^1(S)$, and $\widehat{\mathcal{A}}$ denote the set of unital homomorphisms from $\mathcal{A}$ to $\R, \C$. Note that $\|\varphi\|=1, \varphi\in\widehat{\mathcal{A}}$.

Note that we know $(l^1(S))'=l^\infty(S)$, hence $\widehat{\mathcal{A}}\subset\mathcal{A}'$. Note that we have $\|\varphi\|=1$, hence if we $\varphi\in l^\infty(S)$, we have
\begin{equation*}
    \|\varphi\|_{l^\infty}=1
\end{equation*}
Then for $z\in S, \|z\|\leq 1$, we have $|\varphi(z)|\leq 1$.

\begin{proposition}
    We naturally identify $\widehat{l^1(S)}$ with $Hom(S,\D)$, i.e. $\{\varphi\in l^\infty(S): \|\varphi\|_{l^\infty}=1\}$.
\end{proposition}
\begin{proof}
    Given $f\in\widehat{l^1(S)}$, we know it's multiplicative, unital, hence all these transfer when viewing $\varphi\in l^\infty(S)$. This implies
    \begin{equation*}
        \varphi(\delta_x)\varphi(\delta_y)=\varphi(\delta_{xy})\Rightarrow \varphi(x)\varphi(y)=\varphi(xy)
    \end{equation*}
    Note here $xy$ denotes the operation on $S$ between $x,y$, for example, could be $x+y$. Hence naturally, if $\varphi\in\widehat{l^1(S)}$, $\varphi$ can also be viewed as $\varphi: S\to\D$, and thus is in $l^\infty$, with $|\varphi(s)|\leq 1$.
\end{proof}
\qed

Furthermore, we can identify elements in $\widehat{l^1(S)}$ with the unit disk. Take $S=\N$.
\begin{proposition}
    \begin{equation*}
        \widehat{l^1(\N)}\cong \D
    \end{equation*}
    where $\D$ denotes the unit disk in $\C$.
\end{proposition}
\begin{proof}
    We motivate this by noticing $\N$ is generated by 1, and thus viewing $\varphi\in\widehat{l^1(\N)}$ as $\varphi\in l^\infty(\N)$, we have $\varphi$ is determined by $\varphi(1)$. And denote $\varphi(1)=z_0$, then we have
    \begin{equation*}
        \varphi(n)=z_0^n
    \end{equation*}
    We thus define a map as follows, for $z\in\D$, 
    \begin{equation*}
        z\mapsto \varphi(n)=z^n
    \end{equation*}
    The map is continuous, bijective, and thus a homeomorphism between compact and Hausdorff space.
\end{proof}
\qed

\begin{proposition}
    The standard topology on $\D$ coincides with the weak-* topology on $\widehat{l^1(\N)}$.
    \begin{equation*}
        D_{std}\cong D_{weak-*}
    \end{equation*}
\end{proposition}
\begin{proof}
    We just need to associate an element in $\D$ with a function $\varphi\in\widehat{l^1(\N)}$. And we do this by
    \begin{equation*}
        z\mapsto \sum_{n\in\N}f(n)x^n
    \end{equation*}
    Both maps are continuous, bijective, and between compact and Hausdorff space, hence is a homeomorphism.
\end{proof}

\subsection{On groups}

We let $G$ denote a discrete commutative group, and we see everything above follows, with one extra property.

\begin{proposition}
    We have the following:
    \begin{equation*}
        \widehat{l^1(G)}\cong \T
    \end{equation*}
    where $\T$ denotes the unit circle $\{z\in\C: |z|=1\}$.
\end{proposition}
\begin{proof}
    For $\varphi\in\widehat{l^1(G)}$, we have
    \begin{equation*}
        |\varphi(x\cdot x^{-1})|=|\varphi(e)|=1
    \end{equation*}
    Because $|\varphi(x)|\leq 1, \forall x$, Hence we have
    \begin{equation*}
        |\varphi(x)|=1, \forall x
    \end{equation*}
    Hence we have $\widehat{l^1(G)}$ naturally identifies with $\T$. Like what we described above, we have what is desired.
\end{proof}
\qed

\begin{remark}
    Take $G=\mathbb{Z}$, if we denote $z\in\T$ as $z=e^{2\pi it}$, then we naturally identify with
    \begin{equation*}
        \sum_{n\in\mathbb{Z}}f(m)e^{2\pi int}
    \end{equation*}
    we denote this mapping as $\widehat{f}$, i.e.
    \begin{equation*}
        \widehat{f}(z)=\sum_{m\in\mathbb{Z}}f(m)e^{2\pi int}
    \end{equation*}
    This is the Fourier transform.
\end{remark}

\section{Lecture 5}
Last time, we talked about if we denote $\mathcal{A}=l^1(G)$, equipped with $\|\cdot\|_{l^1}$, under convolution, we have
\begin{equation*}
    \widehat{\mathcal{A}}\text{ ``='' } Hom(G,\T)
\end{equation*}

If we take $G=(\mathbb{Q}, +)$, one can ask the question if $\widehat{\mathcal{A}}$ is big enough. And we will se later in the course, the answer is yes.

For pointwise multiplication, $\widehat{G}$ forms a group, and in fact $\widehat{G}$ is a compact topological group.

For any compact commutative group $G$, for exapmle $\R^n$ under $+$. Define
\begin{equation*}
    \widehat{G}=\text{ continuous homomorphisms into } \T
\end{equation*}
\begin{remark}
    We now require continuous with this general $G$ (previously was not required for discrete group $G$).
\end{remark}

\begin{proposition}
    Let $G$ be a locally compact and commutative group, we have $\widehat{G}$ as a locally compact, commutative group.
\end{proposition}

We define the pairing between $G$ and $\widehat{G}$ as follows: $x\in G, \varphi\in\widehat{G}$,
\begin{equation*}
    \varphi(x)=\langle x, \varphi \rangle
\end{equation*}

And we have the folliwng map is a homeomorphism.
\begin{equation*}
    G\mapsto \widehat{\widehat{G}}
\end{equation*}

Now let $G,H$ denote locally compact groups, and $\phi:G\to H$ bet a continuous homomorphism.
Note we have the following diagram:
\begin{equation*}
    G \xrightarrow{\phi} H
\end{equation*}
\begin{equation*}
    \widehat{G} \xleftarrow{\phi}\widehat{H}
\end{equation*}
If we take an element $\psi\in\widehat{H}$, we consider $\psi\circ\phi$. We get $\psi\circ\phi\in\widehat{G}$.

\begin{definition}[category, functor]
    A category is specified by 
    \begin{enumerate}
        \item a set of objects
        \item morphisms between objects
        \begin{enumerate}
            \item $X, Y, Z$ are objects, and if
            \begin{equation*}
                X\xrightarrow{\Phi} Y\xrightarrow{\Psi} Z
            \end{equation*}
            \item For each object $X$, there is an identity morphism $1_X$.
        \end{enumerate}
    \end{enumerate}
    And a functor is defined to be such a morphism between categories.
\end{definition}
\begin{example}
    For category of finite vector spaces $V$, passing from vector space to its dual $V'$ is a functor.

    Note that we have the following diagram, assuming they are vector spaces over the reals,
    \begin{equation*}
        V\xrightarrow{T}W
    \end{equation*}
    \begin{equation*}
        V'\xLeftarrow{T^t}W'
    \end{equation*}
    \begin{equation*}
        V''\xrightarrow{T^{tt}}W''
    \end{equation*}
\end{example}

The map going in the same directions $V\to W$, and $V''\to W''$ is called covariant, whereas $V'\leftarrow W'$ is called contravariant.
\begin{example}
    For category of locally compact groups $G,H$, assigning the dual group is a functor:
    \begin{equation*}
        G\to H
    \end{equation*}
    \begin{equation*}
        \widehat{G}\leftarrow\widehat{H}
    \end{equation*}
    \begin{equation*}
        \widehat{\widehat{G}}\to\widehat{\widehat{H}}
    \end{equation*}
\end{example}
\begin{example}
Now let $X$ be a compact space. Given $\Phi$ continuous map between $X\to Y$.
\begin{equation*}
    X\xrightarrow{\Phi}Y
\end{equation*}
\begin{equation*}
    C(X)\leftarrow{C(\Phi)} C(Y)
\end{equation*}
For $f\in C(Y)$, we define
\begin{equation*}
    C(\Phi)(f)=f\circ\Phi
\end{equation*}
Similarly, we take
\begin{equation*}
    X\xrightarrow{\varphi}\xrightarrow{\phi}Z
\end{equation*}
\begin{equation*}
    C(X)\xleftarrow{C(\varphi)}C(Y)\xleftarrow{C(\phi)}C(Z)
\end{equation*}
where for $f\in C(Y), C(\varphi)(f)=f\circ\varphi$, and $g\in C(Z), C(\phi)=g\circ \phi$. 
This is a contravariant functor from the category of compact Hausdorff space into the category of unital commutative Banach algebra.
\end{example}

Now we build an important intuition that given a unital algebra homomorphism map between $C(X)$ and $C(Y)$, there eixsts a map from $X$ to $Y$.
\begin{proposition}
    Suppose $X,Y$ are compact, there exists a unital algebra homomorphism
    \begin{equation*}
        C(X)\xleftarrow{F} C(Y)
    \end{equation*}
    Then there exists a continuous homomorphism $\check{F}: X\to Y$.
\end{proposition}
\begin{proof}
    Define $\varphi_x:C(X)\to\C$ as the evalutation map: take $f\in C(X)$,
    \begin{equation*}
        \varphi_x(f)=f(x)
    \end{equation*}
    Then $\varphi_x\circ F\in\widehat{C(Y)}$. And we know that any element in $\widehat{C(Y)}$ is a point evaluation, i.e. there exists $y\in Y$ such that
    \begin{equation*}
        \varphi_y=\varphi_x\circ F
    \end{equation*}
    We thus define $\check{F}(x)=y$ as such that it satisfies the above equation. We need to show $\check{F}$ is continuous. Note that $X,Y$ are compact Hausdorff spaces, and the topology on $Y$ is the coarest topology making all functions $g\in C(Y)$ continuous.
    \begin{align*}
        g\circ\check{F}(x)&=g(\check{F}(x))\\
        &=g(y: \varphi_y=\varphi_x\circ F)\\
        &=\varphi_y(g: \varphi_y=\varphi_x\circ F)\\
        &=\varphi_x\circ F(g)\\
        &=F(g)(x)
    \end{align*}
    Hence by $F,g$ being continuous, we have $\check{F}$ is also continuous. 
\end{proof}
\qed

There is a natural bijection between the continuous functiosn from $X$ to $Y$, and the unital algebra homomorphism from $C(X)$ to $C(Y)$.

A quick reminder:

\begin{remark}
    For $X$ compact, the weak-* topology coincides with the standard topology.
\end{remark}

\section{Lecture 6}

Now we begin. From Aren "not talking to you is torture."

Let $\mathcal{A}$ be a unital Banach algebra.

We write $GL_n(\mathcal{A})$ to denote the general linear group, the group formed by $n\times n$ matrices with entries from $\mathcal{A}$. 

The less standard notation is $GL_I(\mathcal{A})$ is the group of invertible elements in $\mathcal{A}$. As we have shown previously, this is a closed subset of $\mathcal{A}$. This is the notation that we will use.

\begin{remark}
    It is easy to see that the product is jointly continuous.
\end{remark}

\begin{proposition}
    The following map is continuous.
    \begin{equation*}
        a\mapsto a^{-1}
    \end{equation*}
\end{proposition}
\begin{proof}
    Given $\|a-b\|<\delta$, we would like to show $\|a^{-1}-b^{-1}\|<\epsilon$. We first rewrite
    \begin{equation*}
        a^{-1}-b^{-1}=a^{-1}(b-a)b^{-1}
    \end{equation*}
    Hence we have
    \begin{equation*}
        \|a^{-1}-b^{-1}\|\leq\|a^{-1}\|\|b-a\|\|b^{-1}\|
    \end{equation*}
    Take $\delta=\epsilon/\|a^{-1}\|\|b^{-1}\|$ would suffice.
\end{proof}
\qed

\begin{proposition}
    Fix $a\in GL(\mathcal{A})$, there exists a neighborhood $O$ of $a$ and a constant $K$ such that for all $y\in O$, we have
    \begin{equation*}
        \|c^{-1}\|<K
    \end{equation*}
\end{proposition}
\begin{proof}
    Let $V=\{d\in\mathcal{A}: \|1-d\|<1/2\}$, then $d$ is invertible and
    \begin{equation*}
        d^{-1}=\sum_{n=0}^\infty(1-d)^n
    \end{equation*}
    We thus have
    \begin{equation*}
        \|d^{-1}\|\leq\frac{1}{1-\|1-d\|}\leq \frac{1}{1-1/2}=2
    \end{equation*}
    We then identify what our $O$ should be. Let $O=aV$, then we want to show that every $ad$ has an inverse with bounded norm. Because $a,d$ are both invertible, $ad$ is also invertible.
    \begin{equation*}
        \|(ad^{-1})\|=\|d^{-1}a^{-1}\|\leq \|d^{-1}\|\|a^{-1}\|\leq 2\|a^{-1}\|
    \end{equation*}
\end{proof}
\qed

\begin{remark}
    For each invertible element, we can find a neighborhood of invertible elements around it, and using that $(1-d)$ is bounded, then $d$ is invertible, we can bound $\|d^{-1}\|$.
\end{remark}

\begin{definition}
    Fix $a\in\mathcal{A}$, the resolvent set of $\mathcal{A}$ is the complement of spectrum of $\mathcal{A}$, i.e. it is the set
    \begin{equation*}
        \{\lambda\in\mathbb{F}: a-\lambda I \text{ is invertible } \}
    \end{equation*}
\end{definition}

Hence the resolvent set is an open, unbounded suset of $\C$ or $\R$.

\begin{definition}[Resolvent function]
    On the resolvent set, $\{\lambda\in\mathbb{F}: a-\lambda1 \text{ is invertible }\}$ is as follows:
    \begin{equation*}
        R(a,\lambda)=(\lambda1_\mathcal{A}-a)^{-1}
    \end{equation*}
    note that $a$ is fixed, and $\lambda$ is the variable here.
\end{definition}

Now we note that this $R_a(\lambda)$ function is nicely behaved.
\begin{proposition}
    The resolvent function $R_a(z)$ is analytic on the resolvent set, and vanishes as $z\to\infty$.
\end{proposition}
\begin{proof}
    We first define the notation of analyticity on an open subset of $\R, \C$: this means for every point in the open set $O$, we can find a power series expansion of the function such that its radius of convergence $>0$.

    Fix $z_0$ in the resolvent set. We know $z_01_\mathcal{A}-a$ is invertible. We consider $(z1_\mathcal{A}-a)$, for $z$ in the resolvent set. We will omit the $1_\mathcal{A}$ for simplicity.
    \begin{equation*}
        z1_\mathcal{A}-a=(z_0-a)-(z_0-z)=(z_0-a)\left(1_\mathcal{A}-\frac{z_0-z}{z_0-a}\right)
    \end{equation*}
    We know the the latter term is invertible if $\|\frac{z_0-z}{z_0-a}\|<1$ has norm, hence we have
    \begin{equation*}
        (z-a)^{-1}=\sum_{n=0}^\infty\left(\frac{z_0-z}{z_0-a}^n \right)(z_0-a)^{-1}
    \end{equation*}

    What happens when we let $z\to\infty$, we consider $R_a(1/z)$, and let $z\to 0$. Note that we have the following:
    \begin{equation*}
        R_a\left(\frac{1}{z}\right)=\left(\frac{1}{z}-a\right)^{-1}=\left(\frac{1-az}{z} \right)^{-1}=z(1-az)^{-1}
    \end{equation*}
    Let $z\to 0$ makes $R_a(1/z)$ go to zero.
\end{proof}
\qed

Now given that $R_a(z)$ is analytic and bounded at $\infty$, we can state the following important theorem.
\begin{theorem}[Nonemptyness of spectrum]
    Let $\mathcal{A}$ be a unital Banach algebra over $\C$, then for any $a\in\mathcal{A}$, we have $\sigma(a)\neq\emptyset$.
\end{theorem}
\begin{proof}
    Assume there exists $a\in\mathcal{A}$, such that $\sigma(a)=\emptyset$.
    If $\mathcal{A}=\mathcal{C}$, then we would have $R_a(\lambda)$ be a bounded entire, complex-valued function defined on all of $\C$. By Liouville's theorem, we must have $R_a(z)$ a constant function, but we know $z\to\infty$, $R_a\to 0$, hence $R_a(z)$ is constantly 0, but this cannot be true. 

    If our $\mathcal{A}$ is a more general Banach algebra, then we take a slight detour of creating an entire bounded function, via the following map
    \begin{equation*}
        z\mapsto \phi(R_a(z))
    \end{equation*}
    where $\phi$ is some nonzero element in $\mathcal{A}'$, guaranteed by Hahn-Banach theorem. Then we have the above map is complex-valued, entire, bounded at $\infty$. Again, the function is constantly 0. 
\end{proof}

With the nonemptyness of spectrum theorem, we now state the Gelfand-Mazur theorem.
\begin{theorem}[Gelfand-Mazur]
    Let $\mathcal{A}$ be a unital Banach algebra over $\C$, if any nonzero element of $\mathcal{A}$ is invertible, then $\mathcal{A}$ is isomorphic to $\C$.
\end{theorem}
\begin{proof}
    For any $a\in\mathcal{A}$, we know $\sigma(a)\neq\emptyset$, hence there exists $\lambda$ such that $\lambda1_\mathcal{A}-a$ is invertible, i.e. $a=\lambda1_\mathcal{A}$, hence establishing an isomorphism between $\mathcal{A}$ and $\C$. In other words, $\mathcal{A}=\C1_\mathcal{A}$.
\end{proof}
\qed


\subsection{Functional Calculus}
\begin{proposition}
    Let $a\in\mathcal{A}$, then if $f(z)=\sum_{n=0}^\infty \alpha_nz^n$ converges for $|z|<r$, where $r>\|a\|$, then $\sum_{n=0}^\infty\alpha_na^n$ converges as well.
\end{proposition}

We first start with proving the following statement.
\begin{lemma}
    Let $f$ be a polynomial, $\mathcal{A}$ is a unital Banach algebra over $\C$, $f=\sum_{n=0}^ka_nx^n$, then for $a\in\mathcal{A}$, we have
    \begin{equation*}
        \sigma(f(a))=f(\sigma(a))
    \end{equation*}
    This states the spectrum of $a$ under $f$ is exactly the spectrum of $f$ evaluated at $a$.
\end{lemma}
\begin{proof}
    $(\Leftarrow)$. We take $\lambda\in\sigma(a)$, and we would like to show $f(\lambda)$ is in the spectrum of $f(a)$. We note that if $\lambda\in\sigma(a)$, then $a=\lambda1_\mathcal{A}$, and $f(\lambda1_\mathcal{A})=f(a)$, hence by definition, $f(a)-f(\lambda)1_\mathcal{A}$ is not invertible implying $f(\lambda)$ is in the spectrum of $f(a)$. Note that this also implies $f(a)-f(\lambda)=(a-\lambda)Q(z)$ for some polynomial $Q(z)$.

    $(\Rightarrow)$. We take $\lambda\in\sigma(f(a))$, i.e. $f(a)=\lambda1_\mathcal{A}$. we would like to show $\lambda=f(y)$, where $y\in\sigma(a)$. If $f$ is some polynomial, then we can rewrite as follows:
    \begin{equation*}
        f(z)-\lambda=d(z-c_1)...(z-c_n)
    \end{equation*}
    Plugging in $a$ we get
    \begin{equation*}
        f(a)-\lambda=d(a-c_11_\mathcal{A})...(a-c_n1_\mathcal{A})
    \end{equation*}
    If $f(a)-\lambda$ is not invertible, then there exists $j$ such that $(a-c_j1_\mathcal{A})$ is not invertible. This implies,
    \begin{equation*}
        c_j\in\sigma(a)
    \end{equation*}
    Recall we would like to show $\lambda=f(y)$, where $y\in\sigma(a)$. In fact, we have $\lambda=f(c_j)$ by knowing $f(c_j)-\lambda=0$. 
\end{proof}
\qed

Now let $f(z)=z^n$, and if $\lambda\in\sigma(a)$, then $\lambda^n\in\sigma(a^n)$ by the previous lemma. Then we know that 
\begin{equation*}
    |\lambda^n|=|\lambda|^n\leq\|a^n\|
\end{equation*}
This implies
\begin{equation*}
    |\lambda|\leq\|a^n\|^{1/n}, \forall n
\end{equation*}
Hence we have
\begin{equation*}
    |\lambda|\leq\liminf_n\{\|a^n\|^{1/n}\}
\end{equation*}

\begin{definition}
    Fix $a\in\mathcal{A}$, we define the spectral radius of $a$, denoted by $r(a)$,
    \begin{equation*}
        r(a)=\sup_\lambda\{|\lambda|:\lambda\in\sigma(a)\}
    \end{equation*}
\end{definition}
\begin{corollary}
    \begin{equation*}
        r(a)\leq\limsup_n\{\|a^n\|^{1/n}\}
    \end{equation*}
\end{corollary}
\begin{proof}
From the previous remark that $|\lambda|\leq\|a^n\|^{1/n}$, hence this follows.
\end{proof}


\section{Lecture 7}
I have not typed up for this?


\section{Lecture 8}
Let $\mathcal{A}$ be a unital Banach algebra. Then for $a\in\mathcal{A}$, and we look at the resolvent of $a$, $R_a(\lambda)$, we've noted that as $\lambda\to\infty$, we have
\begin{equation*}
    \lim_{\lambda\to\infty}R_a(\lambda)=\lim_{\lambda\to\infty}(\lambda1_\mathcal{A}-a)^{-1}=\lim_{\lambda\to\infty}\lambda^{-1}\sum_{n=0}^\infty a^n\lambda^{-n}
\end{equation*}
And the above Laurent series converges for $|\lambda|\geq\|a\|$.

Recall that we define the spectral raidus, $r(a)$, as 
\begin{equation*}
    r(a)=\sup\{|\lambda|:\lambda\in\sigma(a)\}\leq\|a\|
\end{equation*}

Now we would like to prove the following proposition.
\begin{proposition}
    \begin{equation*}
        r(a)=\lim\|a^n\|^{1/n}
    \end{equation*}
\end{proposition}
\begin{proof}

If we let $\lambda=1/z$, then 
\begin{equation*}
    R(a,z)=z\sum_{n=0}^\infty a^nz^n
\end{equation*}
This converges for $|z|\leq\|a\|^{-1}$, but maybe?? also for $|z|< r(a)^{-1}$?

For $r>r(a)$, i.e. $|z|\leq r^{-1}$, we  know $\sum_na^nr^n$ converges for $r>r(a)$. 

know $z\sum a^nz^n$ converges absolutely. In particular,
\begin{equation*}
    a^nz^n\to 0
\end{equation*}
Hence there exists $M$ such that for $n\geq M$, we have
\begin{equation*}
    \|a^nr^{-n}\|\leq 1
\end{equation*}
This implies that
\begin{equation*}
    \|a^n\|\leq r^n \Rightarrow \|a^n\|^{1/n}\leq r 
\end{equation*}
for all $n\geq M$.

This implies that
\begin{equation*}
    \limsup \|a^n\|^{1/n}\leq r
\end{equation*}
And note that $r$ is arbitrary close to the spectral radius $r(a)$. Hence we have
\begin{equation*}
    \limsup \|a^n\|^{1/n}\leq r(a)\leq\liminf\|a^n\|^{1/n}
\end{equation*}
We've derived the second inequality from last class. Hence all inequalities become equalities.
This gives us
\begin{equation*}
    r(a)=\lim \|a^n\|^{1/n}
\end{equation*}

\end{proof}
\qed


For each $\varphi\in\mathcal{A}'$, consider the map
\begin{equation*}
    \lambda\mapsto\lambda^{-1}\sum\varphi(a^n)\lambda^{-n}
\end{equation*}
This series converges for $r>r(a)$.
We can apply the same process, to argue that there exists $M_\varphi$ such that
\begin{equation*}
    \|\varphi(a^n)r^{-n}\|\leq M_\varphi
\end{equation*}
for all $n\geq 0$. Note that $M_\varphi$ could be different for all $\varphi$. 

Note that 
\begin{equation*}
    \mathcal{A}\to \mathcal{A}'\to \mathcal{A}''
\end{equation*}
there is a natural injection of $a\mapsto\widehat{a}\in\mathcal{A}''$.

For each $n$, definite $F_n\in\mathcal{A}''$, by $F_n(\varphi)=|\varphi(a^nr^{-n})|\leq M_\varphi$. Applying the UBP, we have
\begin{equation*}
    |F_n(\varphi)|\leq M\Rightarrow |\varphi(a^n)r^{-n}|\leq M
\end{equation*}
This implies that 
\begin{equation*}
    |\varphi(a^n)|\leq r^nM
\end{equation*}
Note that by Hahn-Banach, for any $b\in\mathcal{A}$, we have
\begin{equation*}
    \|b\|=\sup\{|\varphi(b)|:\|\varphi\|=1\}
\end{equation*}
Taking $n$-th root of both sides, we gets
\begin{equation*}
    \|a^n\|\leq r^nM\Rightarrow \|a^n\|^{1/n}\leq rM^{1/n}\to r
\end{equation*}
Hence we again obtain the same result.

\qed



Recall UBP.
\begin{theorem}[Uniform Boudnedness Principle]
    Let $X$ be Banach, and $Y$ be normed, let $T_n:X\to Y$ be a family of linear operators, and if for all $x\in X$, we have
    \begin{equation*}
        \|T_n(x)\|<\infty
    \end{equation*}
    Then for all $n$, we have
    \begin{equation*}
        \|T_n\|<\infty
    \end{equation*}
\end{theorem}

Note that 
if $\mathcal{A}$ is unital, and if $\mathcal{A}\subset\mathcal{B}$ with some unit. For $a\in\mathcal{A}$, if $a$ is not invertible in $A$, then it might be invertible in $\mathcal{B}$. Hence if we use $\sigma_\mathcal{A}(a)$ to denote the spectrum of $a$ in $\mathcal{A}$.
\begin{proposition}
    \begin{equation*}
        \sigma_\mathcal{B}(a)\subset \sigma_\mathcal{A}(a)
    \end{equation*}
\end{proposition}


\begin{example}
    Let $\mathcal{B}=l^1(\mathbb{Z})$, and let $\mathcal{A}=l^1(\N)$, equipped with convolution. 
    
    Clearly $\mathcal{A}\subset\mathcal{B}$. And note that the delta function at 1, $\delta_1$ is not invertible in $\mathcal{A}$ but it has an inverse $\delta_{-1}$ in $\mathcal{B}$. Hence we see $0\in\sigma_\mathcal{A}(a)$, but $0\not\in\sigma_\mathcal{B}(a)$.


\end{example}

\begin{proposition}[Spectral radius is preserved]
    For $\mathcal{A}\subset\mathcal{B}$, we have
    \begin{equation*}
        r_\mathcal{A}(a)=\lim\|a^n\|^{1/n}=r_\mathcal{B}(a)
    \end{equation*}
\end{proposition}

\begin{proposition}
    Let $X$ be compact, and let $\mathcal{A}=C(X)$. Then for $f\in C(X)$, we have
    \begin{equation*}
        \|f^2\|_\infty=\|f\|_\infty^2
    \end{equation*}
\end{proposition}
\begin{proof}
    Look at where $f$ takes $\|f\|_\infty$, and sqaure it, since when $X$ is compact, you can actually obtain the point where $|f(x)|=\|f\|_\infty$ .
\end{proof}

\begin{remark}
The same property holds for $f$ in any unitla subalgebra of $C(X)$, for example, if $X\subset\C$, and let $\mathcal{A}$=functions that are holomorphic on an open subset of $\C$ that are in $X$.
\end{remark}

Let $\mathcal{A}$ be a unital Banach alagebra with the property such that for any $a\in\mathcal{A}$, we have
\begin{equation*}
    \|a^2\|=\|a\|^2
\end{equation*}
This implies that
\begin{equation*}
    \|a^4\|=\|a\|^4
\end{equation*}
By induction, for any $n$, we have
\begin{equation*}
    \|a^{2^n}\|=\|a\|^{2^n}
\end{equation*}
Hence by taking $1/2^n$-root of both sides, we get that the spectral radius of $r(a)$
\begin{equation*}
    r(a)=\|a^{2^n}\|^{1/2^n}=\|a\|
\end{equation*}

Let $\mathcal{H}$ be a Hilbert space, over $\C$, and let $\mathcal{A}=B(H)$, i.e. the bounded linear operators on $\mathcal{H}$., and equip with the operation of taking adjoint. $T\mapsto T^*$.

\begin{proposition}
    For any $T\in B(\mathcal{H})$, we have
    \begin{equation*}
        \|T^*T\|=\|T\|^2
    \end{equation*}
\end{proposition}
\begin{proof}
    We know that $\|T^*\|=\|T\|$. And thus
    \begin{equation*}
        \|T^*T\|\leq\|T^*\|\|T\|=\|T\|^2
    \end{equation*}
    For the reverse direction, let $\xi\in\mathcal{H}$, then
    \begin{equation*}
        \|T(\xi)\|^2=\langle T\xi, T\xi\rangle=\langle \xi, T^*T\xi\rangle\leq \|T^*T\|\|\xi\|^2
    \end{equation*} 
    where the last inequality follows form Cauchy-Schwartz. This implies that
    \begin{equation*}
        \|T(\xi)\|\leq \|T^*T\|^{1/2}\|\xi\|
    \end{equation*}
    which by definition, gives
    \begin{equation*}
        \|T\|\leq\|T^*T\|^{1/2}
    \end{equation*}
    Taking squares we get the desired result.
\end{proof}
\qed

\begin{corollary}
    If $T^*=T$, then
    \begin{equation*}
        \|T^2\|=\|T\|^2
    \end{equation*}
    And we have
    \begin{equation*}
        r(T)=\|T\|
    \end{equation*}
    where the spectral radius is determined by the algebra elements.
\end{corollary}

Note that for general $T$, we have  $T^*T$ is always self-adjoint,
\begin{equation*}
    \|T\|^2=\|T^*T\|=r(T^*T)
\end{equation*}
Then we have
\begin{equation*}
    \|T\|=(r(T^*T))^{1/2}
\end{equation*}
where the spectral radius is determined by the *-algebra structure.